\documentclass{article}

\usepackage{amsmath}
\usepackage{amssymb}
\usepackage[spanish]{babel}
\usepackage[margin=1.5in]{geometry}
\usepackage{graphicx}
\usepackage[utf8]{inputenc}

\renewcommand{\Bbb}{\mathbb}

\begin{document} 

\section{EDO 1}

Una función escalar es diferenciable si y sólo si se puede escribir lo siguiente:

\begin{equation}
f(x+h) - f(x) = f'(x) h + \mu(h) h
\end{equation}

En el lado derecho de esta igualdad, el primer sumando es el \textbf{diferencial total de la función}, y el segundo es un infinitésimo que tiende a cero cuando h tiende a cero.

Propiedades:

\begin{subequations}
\begin{align}
d k & = 0 \\
d(k f) & = k df \\
d(f \pm g) & = df \pm dg \\
d(f g), d( f / g ) & = \textbf{igual que derivada} \\
dy & = f'(x) h \wedge h = dx \Longrightarrow y' = \frac{dy}{dx} = f'(x)
\end{align}
\end{subequations}

Esta última se conoce como la notación de Leibnitz.

Una ecuación diferencial, o ED, es una ecuación donde las incógnitas son funciones; toda ED relaciona una función a determinar, su(s) variable(s) independiente(s), y las derivadas de la función. Las EDs pueden ser:

//TODO Usar bullet list
* Ordinarias: La función incógnita tiene una sola variable independiente.

* En derivadas parciales: La función incógnita tiene más de una variable independiente.


\subsection{Orden de una ED}

El orden de una ED está dado por la derivada de mayor orden de la función incógnita.

\subsection{Expresión general}

Expresión general orden 1:

\begin{equation}
F(x, y, y') = 0
\end{equation}

Normalizada orden 1:

\begin{equation}
y' = g(x, y)
\end{equation}

Expresión general para orden N:

\begin{equation}
F(x, y, y', y'', ..., y^(n)) = 0
\end{equation}

Se dice que $x, y, y', ..., y^(n)$ son las variables de la ED $F$.

\subsection{Grado de una ED}

Aquellas ED que pueden expresarse como polinomios respecto al orden de las derivadas, y donde además los coeficientes que multiplican a las derivadas son constantes o funciones de $x$, tienen \textbf{grado}. El mismo corresponde al exponente al que está elevada la derivada de mayor orden.

Si algún coeficiente depende de $y$, la ED no tiene grado.

\subsection{Soluciones de una ED}

Sea $y = g(x)$ una función definida en un intervalo $I$, con $n$ derivadas continuas en dicho intervalo. Si al reemplazar $y$ en una ED, la misma se reduce a una identidad, se dice que $y = g(x)$ es solución de la ED en $I$. En tal caso, $I$ es denominado intervalo de solidez, intervalo de existencia, o dominio de la solución.

\subsubsection{Solución general (SG)}

La SG de una ED de orden $n$ es una relación funcional entre sus variables 
que contiene $n$ constantes arbitrarias linealmente independientes.

\subsubsection{Solución particular (SP)}

Se obtiene a partir de la SG al darle valores concretos a sus constantes; usualmente, ello requiere $n$ valores iniciales o condiciones de contorno de la función incógnita y sus derivadas.

\subsubsection{Solución singular (SS)}

Una función es SS de una ED cuando la reduce a una identidad, pero no puede obtenerse a partir de una SG. No toda ED tiene SS.

\end{document} 
