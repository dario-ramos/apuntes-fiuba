\documentclass{article}

\usepackage{amsmath}
\usepackage{amssymb}
\usepackage{enumerate}
\usepackage[spanish]{babel}
\usepackage{cancel}
\usepackage{caption}
\usepackage[margin=1.5in]{geometry}
\usepackage{graphicx}
\usepackage[utf8]{inputenc}
\usepackage{tcolorbox}
\usepackage{esint}
\usepackage{hyperref}
\hypersetup{
    colorlinks,
    citecolor=black,
    filecolor=black,
    linkcolor=black,
    urlcolor=black,
}

\renewcommand{\Bbb}{\mathbb}

\tcbuselibrary{theorems}

\title{Ejercicios de Análisis Matemático II A (63.01) \\
Guía 3 - Diferenciabilidad, tangentes \\
Cátedra Acero \\
1° C 2004}
\author{Darío Eduardo Ramos}

\begin{document}
\maketitle

\tableofcontents{}
\newpage

\section*{3.1}
\label{sec:3.1}
\addcontentsline{toc}{section}{\nameref{sec:3.1}}

\textbf{En los siguientes casos:} 

\begin{enumerate}
\bfseries

\item Calcular el gradiente de $f$ en el punto $P_0 = (x_0, y_0)$
\item En un gráfico en 3D, trazar aproximadamente la curva de nivel $C$ de $f$ que pasa por $P_0$, su recta tangente en $P_0$ y el gradiente de $f$ con origen en $P_0$.
\item (En el mismo gráfico) Ubicar el punto $Q_0$ de la superficie $z = f(x,y)$ cuya proyección en el plano $xy$ es $P_0$; dibujar aproximadamente la superficie $z$ en la vecindad de $Q_0$ y su plano tangente.
\item (En el mismo gráfico) Marcar sobre la superficie $z$ la curva cuya proyección en el plano $xy$ es $C$.
\item (En el mismo gráfico) Graficar el vector $N = (f'_x(P_0), f'_y(P_0), -1)$ con origen en $Q_0$.
\item ¿Qué relación hay entre los distintos objetos geométricos del gráfico? Explicar las relaciones con palabras y ponerlas en evidencia en los dibujos.

\end{enumerate}

\begin{enumerate}[(a)]
\bfseries

\item $f(x,y)=x^2+y^2, P_0 = (1,1)$
\item $f(x,y)=\sqrt{4-x^2-y^2}, P_0 = (1,-1)$
\item $f(x,y)=5+2x-3y, P_0=(0,0)$
\item $f(x,y)=7+xy, P_0=(0,0)$
\item $f(x,y)=7+x^2-y^2, P_0=(1,1)$
\item $f(x,y)=\sqrt{x^2+y^2/4}, P_0=(0,2)$

\end{enumerate}
\hrule

\subsection*{3.1.a}
\label{subsec:3.1.a}
\addcontentsline{toc}{subsection}{\nameref{subsec:3.1.a}}

\subsubsection*{3.1.a.1}
\label{subsubsec:3.1.a.1}
\addcontentsline{toc}{subsubsection}{\nameref{subsubsec:3.1.a.1}}

Por simple inspección, $f \in C^1$ y por ende existe el gradiente:

\begin{equation}
\overrightarrow{ \nabla f }(x,y) = (f_x(x,y), f_y(x,y)) = (2x, 2y)
\end{equation}

Ergo:

\begin{equation}
\tcboxmath[colback=orange!25!white,colframe=orange,title=3.1.a.1]
{
\overrightarrow{ \nabla f }(x_0,y_0) = (2, 2)
}
\end{equation}

\end{document}
