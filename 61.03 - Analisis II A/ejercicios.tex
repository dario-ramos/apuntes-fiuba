\documentclass{article}

\usepackage{amsmath}
\usepackage{amssymb}
\usepackage{enumerate}
\usepackage[spanish]{babel}
\usepackage{cancel}
\usepackage{caption}
\usepackage[margin=1.5in]{geometry}
\usepackage{graphicx}
\usepackage[utf8]{inputenc}
\usepackage{tcolorbox}
\usepackage{esint}
\usepackage{hyperref}
\hypersetup{
    colorlinks,
    citecolor=black,
    filecolor=black,
    linkcolor=black,
    urlcolor=black,
}

\renewcommand{\Bbb}{\mathbb}

\tcbuselibrary{theorems}

\title{Ejercicios de Análisis Matemático II A (63.01) \\ Cátedra Acero \\ 1° C 2004}
\author{Darío Eduardo Ramos}

\begin{document}
\maketitle

\tableofcontents{}
\newpage

\section{Geometría del plano y el espacio}

\hrule
\vspace{10 pt}
1. Hallar en cada caso la ecuación implícita de un plano que satisfaga las condiciones dadas. Analizar si esas condiciones determinan el plano unívocamente. 

\begin{enumerate}[(a)]
\item pasa por $(1, 1, 0)$, $(0, 2, 1)$ y $(3, 2, -1)$;

\item pasa por $(2, 0, 1)$ y es perpendicular a la recta que pasa por $(1, 1, 0)$ y $(4, -1, -2)$;

\item contiene a la intersección de los planos de ecuaciones $x + y - 2z = 0$ y $2x - y + z = 2$;

\item es perpendicular al plano de ecuación $2x + 3y + 4z = 5$ y contiene a la recta de ecuaciones $x + y = 2$, $y - z = 3$.
\end{enumerate}
\hrule

\begin{enumerate}[a)]

\item En $\Bbb R^n$, $n$ puntos distintos determinan unívocamente un plano $\Leftrightarrow$ no son colineales. Concretamente, no están sobre la misma recta; esto equivale a exigir que los vectores que los unen no son paralelos. En este caso:

\begin{subequations}
\begin{align}
& P = (1, 1, 0) \wedge Q = (0, 2, 1) \wedge R = (3, 2, -1) \\
& \overline{PQ} = Q - P = (0-1, 2-1, 1-0) = (-1, 1, 1) \\
& \overline{QR} = R - Q = (3-0, 2-2, -1-1) = (3, 0, -2)
\end{align}
\end{subequations}

Estos vectores no son múltiplos entre sí, y por ende no son colineales, y el plano que determinan es único. Para calcular la expresión implícita de dicho plano, vale decir:

\begin{subequations}
\begin{align}
a (x-x_0) + b (y-y_0) + c (z-z_0) = 0 \\
a x + b y + c z + d = 0, d = -a x_0 -b y_0 -c z_0
\end{align}
\end{subequations}

Nótese que la ecuación del plano es un producto escalar igualado a cero (con una constante de ajuste). Ergo, el vector $(a,b,c)$ es ortogonal al plano. Por lo tanto, puede obtenerse haciendo el producto vectorial entre dos vectores no colineales pertenecientes al plano, como lo son $\overline{PQ}$ y $\overline{QR}$:

\begin{subequations}
\begin{align}
(a,b,c) &= \overline{PQ} \times \overline{QR} = \det \begin{vmatrix}
\hat{i} & \hat{j} & \hat{k} \\
-1 & 1 & 1 \\
3 & 0 & -2
\end{vmatrix} = \hat{i} \begin{vmatrix}1 & 1 \\ 0 & -2\end{vmatrix} - \hat{j} \begin{vmatrix}-1 & 1 \\ 3 & -2\end{vmatrix} + \hat{k} \begin{vmatrix}-1 & 1 \\ 3 & 0\end{vmatrix} \\
& = \hat{i} (-2-0) -\hat{j} (2-3) +\hat{k} (0-3) = (-2, 1, -3)
\end{align}
\end{subequations}

Obtenido el vector normal $(a,b,c)$, reemplazando cualquier vector perteneciente al plano como $(x_0, y_0, z_0)$, se obtiene la ecuación implícita buscada. Por ejemplo, con $(x_0, y_0, z_0) = P = (1, 1, 0)$:

\begin{equation}
-2 (x-1) + 1 (y-1) - 3 (z-0) = 0 \Rightarrow -2x +2 +y -1 -3z = 0 
\end{equation}

\begin{equation}
\tcboxmath[colback=orange!25!white,colframe=orange, title=Plano único]
{ -2x +y -3z + 1 = 0 }
\end{equation}

\item Dadas una recta, hay infinitos planos perpendiculares a ella. Sin embargo, especificando un punto del plano, incluso si es sobre la recta, determina unívocamente el plano. Por lo tanto, en este caso también resulta único el plano.

Si la recta es perpendicular al plano, su vector dirección sirve como vector normal del plano, o sea $(a,b,c)$ en la ecuación implícita del plano. Para obtener el vector dirección de una recta, basta con restar dos puntos de la misma:

\begin{equation}
(a, b, c) = (1, 1, 0) - (4, -1, -2) = (-3, 2, 2)
\end{equation}

Y dado que también se tiene un punto del plano, reemplazándolo como $(x_0, y_0, z_0)$ se obtiene de forma directa la ecuación del plano:

\begin{subequations}
\begin{align}
& a (x-x_0) + b (y-y_0) + c(z-z_0) = 0 \\
& -3 (x-2) + 2 (y-0) + 2 (z-1) = 0 \Rightarrow -3x + 2y + 2z +6-2 = 0
\end{align}
\end{subequations}

\begin{equation}
\tcboxmath[colback=orange!25!white,colframe=orange, title=Plano único]
{ -3x + 2y +2z + 4 = 0 }
\end{equation}

\item La intersección de dos planos puede ser vacía, una recta o un plano (si son el mismo plano). Ergo, la única forma de que la intersección de dos planos determine unívocamente un plano es que los dos planos sean el mismo. En todo caso, resolver el sistema de ecuaciones dará la respuesta.

\begin{subequations}
\begin{align}
\left\{
\begin{array}{ll}
x + y -2z = 0 \\
2x -y + z = 2
\end{array}
\right. \overset{R_2 \leftarrow -2R_1 + R_2}{\Longrightarrow}
\left\{
\begin{array}{ll}
x & + y -2z = 0 \\
  & - 3y + 5z = 2
\end{array}
\right.
\end{align}
\end{subequations}

Hay dos ecuaciones l.i. entre sí, y 3 incógnitas. Por lo tanto, hay un grado de libertad y la solución es una recta. La expresión paramétrica de la misma puede hallarse, por ejemplo, despejando $z$ en función de $y$ en $R_2$ y reemplazando en $R_1$ para hallar $x$ en función de $y$:

\begin{subequations}
\begin{align}
& R_2: 5z = 2 + 3y \Rightarrow z = \frac{2}{5} + \frac{3}{5} y \\
& R_1: x + y - 2 \left( \frac{2}{5} + \frac{3}{5} y \right) = 0 \Rightarrow x + y -\frac{4}{5} - \frac{6}{5} y = 0 \Rightarrow x = \frac{4}{5} + \frac{1}{5} y
\end{align}
\end{subequations}

Pensando en $y$ como el parámetro $\lambda$, la intersección buscada es la recta $R$:

\begin{subequations}
\begin{align}
& R: \left( \frac{4}{5} + \frac{1}{5} \lambda, \lambda, \frac{2}{5} + \frac{3}{5} \lambda \right) \\
& R: \left(\frac{1}{5}, 1, \frac{3}{5} \right) \lambda + \left(\frac{4}{5}, 0, \frac{2}{5} \right) \\
& R: (1, 5, 3) t + (4, 0, 2)
\end{align}
\end{subequations}

Por la recta $R$, pasan infinitos planos. Con elegir un punto fuera de la recta, alcanza para determinar uno cualquiera. Con que tenga un valor distinto de $5$ en la componente $y$ es suficiente, por lo tanto se elige $(0, 1, 0)$. Tomando ese punto y dos puntos de la recta, los de $t=0$ y $t=1$, se tienen 3 puntos para determinar un plano de todos los posibles:

\begin{subequations}
\begin{align}
& P = (0, 1, 0) \wedge Q = (4, 0, 2) \wedge R = (5, 5, 5) \\
& \overline{PQ} = Q - P = (4-0, 0-1, 2-0) = (4, -1, 2) \\
& \overline{QR} = R - Q = (5-4, 5-0, 5-2) = (1, 5, 3) \\
& (a, b, c) = \overline{PQ} \times \overline{QR} = \begin{vmatrix}
\hat{i} & \hat{j} & \hat{k} \\
4 & -1 & 2 \\
1 & 5 & 3
\end{vmatrix} = \hat{i} \begin{vmatrix}
-1 & 2 \\
5 & 3
\end{vmatrix} -\hat{j} \begin{vmatrix}
4 & 2 \\
1 & 3
\end{vmatrix} +\hat{k} \begin{vmatrix}
4 & -1 \\
1 & 5
\end{vmatrix} = \\
& (a, b, c) = (-3-10, -(12-2), 20-(-1)1) = (-13, -10, 21) \\
& -13 (x-0) -10 (y-1) +21 (z-0) = 0
\end{align}
\end{subequations}

\begin{equation}
\tcboxmath[colback=orange!25!white,colframe=orange, title=Un plano de infinitos posibles]
{ -13x - 10y + 21z + 10 = 0 }
\end{equation}

\item Sean:

\begin{subequations}
\begin{align}
& \Pi: 2x + 3y + 4z = 5 \text{ el plano dato} \\
& R: \left\{ \begin{array}{ll}
x + y = 2 \\
y - z = 3
\end{array} \right. \text{ la recta que debe estar contenida en } \Phi \\
& \Phi: \text{ El plano a determinar, perpendicular a } \Pi \text{, contiene a R}
\end{align}
\end{subequations}

Para que $\Phi$ contenga a $R$ y a la vez sea perpendicular a $\Pi$, es necesario que $R$ sea paralela a la normal del plano. Para determinar el vector dirección de $R$, se la pasa a su forma paramétrica:

\begin{subequations}
\begin{align}
& R_1: x = 2 - y \\
& R_2: -z = 3 - y \Rightarrow z = -3 + y \\
& R: (2-\lambda, \lambda, -3+\lambda) = (-1, 1, 1) \lambda + (2, 0, -3) \\
& \overline{v_R} = R(t=0) - R(t=1) = (2, 0, -3) - (1, 1, -2) = (1, -1, -1)
\end{align}
\end{subequations}

Dado que $R$ no es paralela a la normal de $\Pi$ (porque $(2, 3, 4)$ no es paralelo a $(1, -1, -1)$), no existe un plano perpendicular a $\Pi$ que contenga por completo a $R$. 

\end{enumerate}

\end{document}