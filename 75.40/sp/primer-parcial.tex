\documentclass[answers]{exam}

\usepackage[linesnumbered,ruled,vlined]{algorithm2e}
\usepackage{amsmath}
\usepackage[spanish]{babel}

\begin{document}

\firstpageheader{75.40 - Algoritmos y Programación I}{Primer cuatrimestre 2004}{Curso 05}

\textbf{Primer Parcial - Tema 03}

\begin{questions}

\question Escribir un algoritmo para leer un entero \textbf{N} y dos \textbf{arreglos A y B} de dimensión \textbf{N} y generar un tercer arreglo \textbf{C} de dimensión \textbf{N} que cumpla las siguientes condiciones:

\begin{equation*}
C(i) = 
\left\{
\begin{array}{ll}
B(i)^i \text{ si } A(i) < B(i) \\
A(i)^i \text{ si } A(i) \geq B(i)
\end{array}
\right., 0 < i < N+1
\end{equation*}

\begin{parts}
\part Indicar el valor mínimo de cada uno de los tres arreglos y las posiciones que ocupan en cada uno de ellos.
\part Ordenar el arreglo \textbf{C} por selección.
\end{parts}

\renewcommand{\solutiontitle}{\noindent\textbf{Solución:}\enspace}

\SetKw{Array}{array}
\SetKw{Begin}{begin}
\SetKw{Const}{const}
\SetKw{Do}{do}
\SetKw{Else}{else}
\SetKw{End}{end}
\SetKw{For}{for}
\SetKw{If}{if}
\SetKw{Integer}{integer}
\SetKw{Of}{of}
\SetKw{Program}{program}
\SetKw{Procedure}{procedure}
\SetKw{Real}{real}
\SetKw{Then}{then}
\SetKw{To}{to}
\SetKw{Type}{type}
\SetKw{Uses}{uses}
\SetKw{Var}{var}

\begin{solution}
En Pascal:

\begin{algorithm}[H]
\Program{} ej1\;
\Uses crt, math\;
\Const max = 100\;
\Type tvec = \Array [1..max] \Of \Real\;
\Var n, i, nMinA, nMinB, nMinC: \Integer; a,b,c: tvec; minA, minB, minC: \Real\;

\BlankLine

\Procedure{} LeerEntero(\Var n: \Integer)\;
\Begin\

\qquad	writeln(`Ingrese un entero:')\;
\qquad	readln(n)\;
\End\;

\BlankLine

\{Ademas de leer los datos, calcula el min y su posicion\}\

\Procedure LeerVector(n: \Integer; \Var a: tvec; \Var minA: \Real; \Var nMinA: \Integer)\;
\Var i: \Integer\;
\Begin\

\qquad	minA := maxInt\;
\qquad	nMinA := -1\;
\qquad	writeln(`Ingrese ', n, ` enteros separados por fin de línea (enter):')\;
\qquad	\For i:=1 \To n \Do\

\qquad	\Begin\

\qquad	\qquad	readln(a[i])\;
\qquad	\qquad	\If (a[i] $<$ minA) \Then\

\qquad	\qquad	\Begin\

\qquad	\qquad	\qquad	minA := a[i]\;
\qquad	\qquad	\qquad	nMinA := i\;
\qquad	\qquad 	\End\;
\qquad	\End\;
\End\;

\BlankLine

\end{algorithm}

\begin{algorithm}[H]
\setcounter{AlgoLine}{27}
\{Ademas de calcular C, calcula su min y posicion\}\

\Procedure GenerarC(n: \Integer; a,b: tvec; \Var c: tvec; \Var minC: \Real; \Var nMinC: \Integer)\;
\Var i: \Integer\;
\Begin\

\qquad	minC := maxInt\;
\qquad	nMinC := -1\;
\qquad	\For i:=1 \To n \Do\

\qquad	\Begin\

\qquad	\qquad	\If a[i] $<$ b[i] \Then c[i] := power(b[i], i) \

\qquad	\qquad	\Else c[i] := power(a[i], i)\;
\qquad	\qquad	\If c[i] $<$ minC \Then\

\qquad	\qquad	\Begin\

\qquad	\qquad	\qquad	minC := c[i]\;
\qquad	\qquad	\qquad	nMinC := i\;
\qquad	\qquad	\End\;
\qquad	\End\;
\End\;

\BlankLine

\Procedure Swap(\Var a, b:\Real)\;
\Var aux: \Real\;
\Begin\

\qquad	aux := b\;
\qquad	b := a\;
\qquad	a := aux\;
\End\;

\BlankLine

\Procedure OrdSel(n: \Integer; \Var a: tvec)\;
\Var i, j: \Integer\;
\Begin\

\qquad	\For 	i:=1 \To n-1 \Do\

\qquad	\qquad	\For j:=i+1 \To n \Do\

\qquad	\qquad \qquad	\If a[i] $>$ a[j] \Then Swap(a[i], a[j])\;
\End\;

\BlankLine

\Begin \{ ppal \}\

\qquad	clrscr()\;
\qquad	LeerEntero(n)\;
\qquad	LeerVector(n, a, minA, nMinA)\;
\qquad	LeerVector(n, b, minB, nMinB)\;
\qquad	GenerarC(n, a, b, c, minC, nMinC)\;
\qquad	writeln(`El mínimo de A es ', minA:0:2, ` y está en la posición ', nMinA)\;
\qquad	writeln(`El mínimo de B es ', minB:0:2, ` y está en la posición ', nMinB)\;
\qquad	writeln(`El mínimo de C es ', minC:0:2, ` y está en la posición ', nMinC)\;

\BlankLine

\qquad	OrdSel(n, c)\;
\qquad	writeln(`C ordenado:')\;
\qquad	\For 	i:=1 \To n \Do\

\qquad	\qquad	write(c[i]:0:2, ` ')\;
\qquad	writeln('')\;
\End. \{ ppal \}

\end{algorithm}

\end{solution}

\question Elaborar la función Prolijafrase que reciba una cadena de caracteres como parámetro y devuelve otra cadena, donde las palabras estén separadas por un solo espacio. Considerar \textbf{palabra} a toda secuencia de caracteres que no sean espacios, comas o puntos. (Estos caracteres: espacios, comas y puntos se consideran separadores de palabras).

\SetKw{Boolean}{boolean}
\SetKw{Char}{char}
\SetKw{Function}{function}
\SetKw{String}{string}

\begin{solution}
De nuevo utilizando Pascal, y además eligiendo el tipo string de Turbo Pascal para representar las cadenas de caracteres:

\begin{algorithm}[H]
\Function EsEspacio(c: \Char): \Boolean\;
\Begin\

\qquad	EsEspacio := (c = \textsc{\char13} \textsc{\char13}) or (c = \textsc{\char13},\textsc{\char13}) or (c = \textsc{\char13}.\textsc{\char13})\;
\End;

\BlankLine

\Function ProlijaFrase(s: \String): \String\;
\Var i: \Integer ; actualEsEspacio, anteriorEsEspacio: \Boolean\;
\Begin\

\qquad	ProlijaFrase := \textsc{\char13}\textsc{\char13}\;
\qquad	anteriorEsEspacio := true\;
\qquad	\For i:=1 \To Length(s) \Do\

\qquad	\Begin\

\qquad	\qquad	actualEsEspacio := EsEspacio(s[i])\;
\qquad	\qquad	\If actualEsEspacio \Then\

\qquad	\qquad	\Begin\

\qquad	\qquad	\qquad	\If not anteriorEsEspacio \Then ProlijaFrase := ProlijaFrase + \textsc{\char13} \textsc{\char13}\;
\qquad	\qquad	\qquad	anteriorEsEspacio := actualEsEspacio\;
\qquad	\qquad	\qquad	continue\;
\qquad	\qquad	\End\;
\qquad	\qquad	ProlijaFrase := ProlijaFrase + s[i]\;
\qquad	\qquad	anteriorEsEspacio := actualEsEspacio\;
\qquad	\End\;
\End;
\end{algorithm}

\end{solution}

\question Analizar el siguiente algoritmo e indicar si existen errores justificando la respuesta. En caso de ser correcto, realizar un seguimiento con los valores 5 y 4 para las variables de entrada.

\begin{algorithm}
Algoritmo MM;

var a, b: entero;

procedimiento NN(i:entero; var j:entero)

inicio

\qquad	i = i + 10;
	
\qquad	j := j + 10;
	
\qquad	escribir(i, j)
	
fin;

inicio

\qquad	leer(a, b);
	
\qquad	NN(a+b, b-a);
	
\qquad	escribir(a, b)

fin
\end{algorithm}

\begin{solution}
Asumiendo las reglas de sintaxis de Pascal:
\begin{itemize}
\item En la línea 3, falta un punto y coma al final de la misma. Toda declaración de encabezado de función o procedimiento debe terminar con punto y coma.
\item En la línea 5, se está utilizando el signo igual (=) para realizar una asignación. Debería ser el signo de dos puntos seguido de un igual (:=).
\item En las líneas 7 y 12, falta un punto y coma al final de la sentencia.
\item En la línea 11, se está pasando como parámetro actual una expresión (b-a) en lugar de una variable. Esto no compila, porque el parámetro formal j fue declarado como variable en la línea 3. No se puede modificar una expresión.
\end{itemize}
\end{solution}

\end{questions}

\end{document}