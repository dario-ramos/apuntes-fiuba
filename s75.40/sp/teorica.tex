\documentclass{article}

\title{Apuntes teóricos de Algoritmos y Programación I (75.40) \\ Curso García (05) \\ 1° C 2004}
\author{Darío Eduardo Ramos}

\usepackage[spanish]{babel}

\begin{document}
\maketitle
\newpage

En esta materia en particular, no tomé notas, o las mismas no eran sustanciales, por dos motivos:

\begin{itemize}
\item Al menos en el curso que me tocó, el número 5 (cátedra García) en el primer cuatrimestre de 2004, las clases teóricas no hacían mucho más que seguir el libro de cabecera de la materia. El mismo era: \textbf{``Algoritmia, arquitectura de datos y programación estructurada''}, de Gustavo López, editado por Nueva Librería en 2003.
	\begin{itemize}
	\item ISBN-10: 9871104065
	\item ISBN-13: 978-9871104062
	\end{itemize}
\item Esta materia era eminentemente práctica. Se hacían diversos ejercicios y trabajos prácticos, y la mejor forma de aprender a programar es sentándose a programar. Si encuentro el código que escribí por aquel entonces, lo agregaré a esta carpeta.
\end{itemize}

Por otro lado, en el momento de escribir estas notas (fines del año 2022), desde hace varios años que la materia 75.40 en la FIUBA utiliza Python como lenguaje inicial en lugar de Pascal. Ergo, cualquier material que suba hoy estará obsoleto. Sin embargo, por interés histórico y por completitud, transcribiré el temario del libro para que tengan una idea de qué contenidos se vieron en esta materia allá en el 2004.

\begin{itemize}
\item Capítulo I: Introducción y nociones básicas de la programación estructurada.
	\begin{itemize}
	\item Computadora.
	\item Ingeniería de software.
	\item Teorema de la programación estructurada (Böhm y Jacopini).
	\item Análisis del problema.
	\item Diseño y verificación de algoritmos.
	\item Codificación.
	\item Documentación.
	\item Estructura de un programa Pascal.
	\end{itemize}
\item Capítulo II: Tipos de datos simples.
	\begin{itemize}
	\item Tipos de datos numéricos: enteros y reales.
	\item Caracter (char).
	\item Lógico (boolean).
	\item Tipos de dato definidos por el usuario.
	\item Cadena (string).
	\item Sentencias.
	\item Asignación.
	\item Expresiones y operaciones aritméticas.
	\item Operaciones de entrada y salida.
	\item Comentarios en el código.
	\item Un primer programa sencillo.
	\end{itemize}
\item Capítulo III: Procedimientos y funciones.
	\begin{itemize}
	\item Diseño descendente (top-down).
	\item Procedimientos.
	\item Ámbito de los identificadores: variables locales y globales.
	\item Parámetros formales versus parámetros actuales.
	\item Parámetros valor versus parámetros variables.
	\item Funciones.
	\item Parámetros procedimiento y parámetros función.
	\item Funciones declaradas a posteriori (forward declaration).
	\item Ejemplos de código.
	\end{itemize}	
\item Capítulo IV: Tipos de datos estructurados.
	\begin{itemize}
	\item Tipos declarados por el usuario.
	\item Enumeraciones (enums).
	\item Conjuntos (sets).
	\item Arrays: vectores y matrices.
	\item Cadenas de caracteres (strings).
	\item Ejemplos de código.
	\end{itemize}
\item Capítulo V: Métodos de búsqueda.
	\begin{itemize}
	\item Búsqueda secuencial.
	\item Búsqueda binaria.
	\item Métodos de ordenación: burbuja (bubblesort), selección e inserción.
	\item Mezcla de listas (merge).
	\item Ejemplos de código.
	\end{itemize}
\item Capítulo VI: Registros.
	\begin{itemize}
	\item Registros jerárquicos/anidados.
	\item Registros variantes.
	\item Constantes tipo registro.
	\item Ejemplos de código.
	\end{itemize}
\item Capítulo VII: Archivos.
	\begin{itemize}
	\item Concepto general.
	\item Declaración y apertura.
	\item Pasaje de archivos por parámetro.
	\item Procedimientos ASSIGN, RESET, REWRITE y CLOSE.
	\item Procedimientos READ y WRITE y función EOF.
	\item Archivos de texto.
	\item Procedimientos WRITE y WRITELN, READ y READLN.
	\item Función EOLN, procedimiento APPEND.
	\item Archivos de acceso directo (con tipos): operaciones.
	\item Ejemplos de código.
	\end{itemize}
\item Capítulo VIII: Operaciones entre archivos.
	\begin{itemize}
	\item Apareo.
	\item Merge.
	\item Intersección.
	\item Unión.
	\item Ejemplos de código.
	\end{itemize}
\item Capítulo IX: Claves.
	\begin{itemize}
	\item Claves en el manejo de archivos de datos.
	\item Clave compuesta.
	\item Superclave, clave candidata, clave primaria, clave secundaria.
	\item Ejemplos de código.
	\end{itemize}
\item Capítulo X: Índices.
	\begin{itemize}
	\item Índices en el manejo de archivos de datos.
	\item Índices primarios y secundarios.
	\item Ejemplos de código.
	\end{itemize}
\item Capítulo XI: Recursividad.
	\begin{itemize}
	\item Concepto.
	\item Recursividad directa e indirecta.
	\item Condición de corte.
	\item Ejemplos de código.
	\end{itemize}
\item Capítulo XII: Manejo de punteros y memoria dinámica.
	\begin{itemize}
	\item Estructura de la memoria.
	\item Punteros con tipo.
	\item New y Dispose en Pascal.
	\item Desreferenciación.
	\item Apuntando a nada: la palabra reservada nil.
	\item El operador de dirección @.
	\item Punteros sin tipo.
	\item El tipo AnsiString.
	\item La directiva de compilador \$H.
	\item Cadenas finalizadas en nulo.
	\item El tipo array [0...X] of char.
	\item Punteros a función/procedimiento.
	\item Convenciones de llamada (calling conventions).
	\item Ejemplos de código.
	\end{itemize}
\item Capítulo XIII: Unidades de biblioteca.
	\begin{itemize}
	\item Unidades de biblioteca (units) en Pascal.
	\item Generación de un archivo ejecutable Pascal, en un programa que no usa unidades.
	\item Generación de un archivo ejecutable Pascal, en un programa que usa unidades.
	\item Beneficios de usar unidades.
	\item Definiendo una unidad en Pascal: secciones de cabecera, interfaz, implementación e inicialización.
	\item Compilación de la unidad.
	\item Invocación de la unidad.
	\item Compile versus make.
	\item Referencias circulares.
	\item Ejemplos 
	\end{itemize}
\item Apéndice A: Pseudocódigo.
	\begin{itemize}
	\item Descomposición de un problema en planteo, solución, implementación y ejecución.
	\item Procesador, ambiente, acciones primitivas y no primitivas, condiciones, algoritmo.
	\item Transformación del ambiente.
	\item Programación estructurada.
	\item Programación modular.
	\end{itemize}
\item Apéndice B: Diagramas de flujo.
\end{itemize}
 
\end{document}