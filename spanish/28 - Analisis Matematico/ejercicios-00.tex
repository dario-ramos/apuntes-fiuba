\documentclass{article}

\usepackage{amsmath}
\usepackage{enumerate}
\usepackage{hyperref}
\hypersetup{
    colorlinks,
    citecolor=black,
    filecolor=black,
    linkcolor=black,
    urlcolor=black,
}
\usepackage{tcolorbox}
\usepackage{xcolor}

\tcbuselibrary{theorems}

% TODO Remove after completing this document
%\usepackage{amssymb}
%\usepackage[spanish]{babel}
%\usepackage{cancel}
%\usepackage{caption}
%\usepackage[margin=1.5in]{geometry}
%\usepackage{graphicx}
%\usepackage[utf8]{inputenc}
%\usepackage{tcolorbox}
%\usepackage{esint}

%
%\renewcommand{\Bbb}{\mathbb}
%

\title{Ejercicios de Análisis Matemático CBC (28) \\
Práctica 0: Preliminares \\
Cátedra Ruiz, curso 62802 \\
1° C 2003}
\author{Darío Eduardo Ramos}

\begin{document}
\maketitle

\tableofcontents{}

\newpage

\section*{1}
\label{sec:1}
\addcontentsline{toc}{section}{\nameref{sec:1}}

\textbf{Calcule:}

\begin{enumerate}[(a)]
\bfseries

\item $\frac{2}{3} - \left[ -\frac{1}{2} + 1 - \left( -\frac{1}{2} - \frac{5}{12} \right) - 2  \right] -\frac{1}{4} - \left[ -1 -\left(2 \frac{1}{6} - \frac{1}{4} -\right) +\frac{2}{3} \right]$

\item $\frac{2}{5} + (-2) \left[ -\frac{1}{2} + 1 -\left( \frac{1}{5} -\frac{3}{10} \right) -\frac{1}{4} \right]$

\end{enumerate}
\hrule

\subsection*{1.a}
\label{subsec:1.a}
\addcontentsline{toc}{subsection}{\nameref{subsec:1.a}}

Esto es un simple repaso de aritmética básica. Además de operar con fracciones, tema que se da por conocido, para resolver estos ejercicios es recomendable tener en cuenta lo siguiente:

\begin{itemize}
\item El producto tiene prioridad sobre la suma y la resta. Vale decir, a la hora de evaluar una expresión como $2 \frac{1}{6} - \frac{1}{4}$, primero se evalúa el producto. Por lo tanto, dicha expresión equivale a $\frac{1}{3} - \frac{1}{4}$.

\item Siempre conviene evaluar por completo los paréntesis más anidados e ir progresando hacia afuera.
\end{itemize}

Aplicando estas ideas, se obtiene:

\begin{subequations}
\begin{align}
&\frac{2}{3} - \left[ -\frac{1}{2} + 1 \textcolor{blue}{ \underbrace{ - \left( -\frac{1}{2} - \frac{5}{12} \right)}_{-\left(    -\frac{11}{12} \right)}} - 2  \right] -\frac{1}{4} - \left[ -1 \textcolor{blue}{ \underbrace{ -\left(2 \frac{1}{6} - \frac{1}{4} \right) }_{-\left( \frac{1}{12} \right)}} +\frac{2}{3} \right] = \\
&\frac{2}{3} \underbrace{ - \left[ -\frac{1}{2} + 1 \textcolor{blue}{ + \frac{11}{12} } - 2  \right] }_{\textcolor{red}{-\left[ -\frac{7}{12} \right]}} -\frac{1}{4} \underbrace{-\left[ -1 \textcolor{blue}{ -\frac{1}{12} } +\frac{2}{3} \right]}_{\textcolor{red}{-\left[ -\frac{5}{12} \right]}} = \\ 
&\frac{2}{3} \textcolor{red}{ +\frac{7}{12} } -\frac{1}{4} \textcolor{red}{ +\frac{5}{12} } = \tcboxmath[colback=orange!25!white,colframe=orange, title=1.a] { \frac{17}{12} }
\end{align}
\end{subequations}

\subsection*{1.b}
\label{subsec:1.b}
\addcontentsline{toc}{subsection}{\nameref{subsec:1.b}}

Resolviendo con la misma metolodogía que el anterior, se obtiene:

\begin{equation}
\tcboxmath[colback=orange!25!white,colframe=orange, title=1.b] { -\frac{3}{10} }
\end{equation}

\end{document}
