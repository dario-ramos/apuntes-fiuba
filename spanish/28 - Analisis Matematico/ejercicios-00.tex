\documentclass{article}

\usepackage{amsmath}
\usepackage{amssymb}
\usepackage[spanish]{babel}
\usepackage{cancel}
\usepackage{enumerate}
\usepackage{hyperref}
\hypersetup{
    colorlinks,
    citecolor=black,
    filecolor=black,
    linkcolor=black,
    urlcolor=black,
}
\usepackage{tcolorbox}
\usepackage{xcolor}

\tcbuselibrary{theorems}

% TODO Remove after completing this document
%\usepackage{caption}
%\usepackage[margin=1.5in]{geometry}
%\usepackage{graphicx}
%\usepackage[utf8]{inputenc}
%\usepackage{tcolorbox}
%\usepackage{esint}


\renewcommand{\Bbb}{\mathbb}


\title{Ejercicios de Análisis Matemático CBC (28) \\
Práctica 0: Preliminares \\
Cátedra Ruiz, curso 62802 \\
1° C 2003}
\author{Darío Eduardo Ramos}

\begin{document}
\maketitle

\tableofcontents{}

\newpage

\section*{1}
\label{sec:1}
\addcontentsline{toc}{section}{\nameref{sec:1}}

\textbf{Calcule:}

\begin{enumerate}[(a)]
\bfseries

\item $\frac{2}{3} - \left[ -\frac{1}{2} + 1 - \left( -\frac{1}{2} - \frac{5}{12} \right) - 2  \right] -\frac{1}{4} - \left[ -1 -\left(2 \frac{1}{6} - \frac{1}{4} -\right) +\frac{2}{3} \right]$

\item $\frac{2}{5} + (-2) \left[ -\frac{1}{2} + 1 -\left( \frac{1}{5} -\frac{3}{10} \right) -\frac{1}{4} \right]$

\end{enumerate}
\hrule

\subsection*{1.a}
\label{subsec:1.a}
\addcontentsline{toc}{subsection}{\nameref{subsec:1.a}}

Esto es un simple repaso de aritmética básica. Además de operar con fracciones, tema que se da por conocido, para resolver estos ejercicios es recomendable tener en cuenta lo siguiente:

\begin{itemize}
\item El producto tiene prioridad sobre la suma y la resta. Vale decir, a la hora de evaluar una expresión como $2 \frac{1}{6} - \frac{1}{4}$, primero se evalúa el producto. Por lo tanto, dicha expresión equivale a $\frac{1}{3} - \frac{1}{4}$.

\item Siempre conviene evaluar por completo los paréntesis más anidados e ir progresando hacia afuera.
\end{itemize}

Aplicando estas ideas, se obtiene:

\begin{subequations}
\begin{align}
&\frac{2}{3} - \left[ -\frac{1}{2} + 1 \textcolor{blue}{ \underbrace{ - \left( -\frac{1}{2} - \frac{5}{12} \right)}_{-\left(    -\frac{11}{12} \right)}} - 2  \right] -\frac{1}{4} - \left[ -1 \textcolor{blue}{ \underbrace{ -\left(2 \frac{1}{6} - \frac{1}{4} \right) }_{-\left( \frac{1}{12} \right)}} +\frac{2}{3} \right] = \\
&\frac{2}{3} \underbrace{ - \left[ -\frac{1}{2} + 1 \textcolor{blue}{ + \frac{11}{12} } - 2  \right] }_{\textcolor{red}{-\left[ -\frac{7}{12} \right]}} -\frac{1}{4} \underbrace{-\left[ -1 \textcolor{blue}{ -\frac{1}{12} } +\frac{2}{3} \right]}_{\textcolor{red}{-\left[ -\frac{5}{12} \right]}} = \\ 
&\frac{2}{3} \textcolor{red}{ +\frac{7}{12} } -\frac{1}{4} \textcolor{red}{ +\frac{5}{12} } = \tcboxmath[colback=orange!25!white,colframe=orange, title=1.a] { \frac{17}{12} }
\end{align}
\end{subequations}

\subsection*{1.b}
\label{subsec:1.b}
\addcontentsline{toc}{subsection}{\nameref{subsec:1.b}}

Resolviendo con la misma metolodogía que el anterior, se obtiene:

\begin{equation}
\tcboxmath[colback=orange!25!white,colframe=orange, title=1.b] { -\frac{3}{10} }
\end{equation}

\section*{2}
\label{sec:2}
\addcontentsline{toc}{section}{\nameref{sec:2}}

\textbf{Calcule:}

\begin{enumerate}[(a)]
\bfseries

\item $\left[ \frac{1}{4} - \left( \frac{2}{3} - \frac{1}{2} \right)^2 \right]^{-2}$

\item $\left[ \left( 4 - \frac{1}{2} \right)^2 + \left( -3-\frac{1}{2} \right)^2 \right]^{\frac{1}{2}}$

\end{enumerate}
\hrule

\subsection*{2.a}
\label{subsec:2.a}
\addcontentsline{toc}{subsection}{\nameref{subsec:2.a}}

El enfoque del punto anterior sigue valiendo, en particular lo referido a evaluar los paréntesis más anidados primero. Lo que se agrega es lidiar con potencias. Y para ello, es conveniente tener presente que:

\begin{itemize}
\item La potenciación no es distributiva respecto a la suma o resta, pero sí respecto al producto.
\begin{itemize}
\renewcommand{\labelitemii}{$\diamond$}
\item $\left(a + b\right)^c \neq a^c + b^c \quad \forall a,b,c \in \mathbb{R}$
\item $\left(a \cdot b\right)^c = a^c \cdot b^c \quad \forall a,b,c \in \mathbb{R}$
\end{itemize}
\item Elevar a una potencia negativa equivale a elevar el recíproco de la misma potencia negada. Simbólicamente, $a^{-b} = \left( \frac{1}{a} \right)^b$, para todo a y b reales, con $a \neq 0$.
\item Elevar a una potencia de la forma $\frac{1}{n}$, con $n$ entero y $n \neq 0$, equivale a tomar la n-ésima raíz de la base: $a^{\frac{1}{n}} = \sqrt[n]{a}$.
\end{itemize}

Aplicando estos conceptos:

\begin{subequations}
\begin{align}
&\left[ \frac{1}{4} - \textcolor{blue}{ \underbrace{ \left( \frac{2}{3} - \frac{1}{2} \right)^2 }_{\left( \frac{1}{6} \right)^2 }} \right]^{-2} = \\
&\left[ \frac{1}{4} - \textcolor{blue}{ \frac{1}{36} } \right]^{-2} = \left[ \frac{8}{36} \right]^{-2} = \left[ \frac{2}{9} \right]^{-2} = \left[ \frac{9}{2} \right]^2 = \tcboxmath[colback=orange!25!white,colframe=orange, title=2.a] { \frac{81}{4} = 20.25 }
\end{align}
\end{subequations}

\subsection*{2.b}
\label{subsec:2.b}
\addcontentsline{toc}{subsection}{\nameref{subsec:2.b}}

Es muy similar al anterior, pero se agrega un concepto. Al expresar un resultado, no es deseable que haya expresiones radicales en el denominador. Por ello se entiende raíces de números o expresiones. Para resolver eso, se multiplica y divide por la misma expresión radical a eliminar. En este caso, partiendo del resultado final con expresión radical:

\begin{equation}
\sqrt{\frac{49}{2}} = \frac{7}{\sqrt{2}} = \frac{7}{\sqrt{2}} \frac{\sqrt{2}}{\sqrt{2}} = \tcboxmath[colback=orange!25!white,colframe=orange, title=2.b] { \frac{7 \sqrt{2}}{2} \approx 4.9497 }
\end{equation}

\section*{3}
\label{sec:3}
\addcontentsline{toc}{section}{\nameref{sec:3}}

\textbf{Calcule:}

\begin{enumerate}[(a)]
\bfseries

\item $ \frac{3^4 \cdot 3^7}{3^{12}} $

\item $\sqrt{ \frac{ (5 \cdot 10^{-6}) (4 \cdot {10}^2) }{8 \cdot {10}^5} }$

\item $ {81}^{\frac{3}{4}} + \left( \frac{16}{49} \right)^{-\frac{1}{2}} + \left( \frac{64}{27} \right)^{\frac{2}{3}} + {32}^{-\frac{4}{5}} + \left( {{2}^{-6}} \right) ^{\frac{2}{3}} + 3^{\frac{7}{2}} \cdot 3^{\frac{1}{2}} $

\item $ \sqrt{5^2} + \sqrt{(-3)^2} + \sqrt[4]{(-9)^2} + \sqrt[3]{-\frac{8}{27}} + \sqrt[3]{81} $

\end{enumerate}
\hrule

\subsection*{3.a}
\label{subsec:3.a}
\addcontentsline{toc}{subsection}{\nameref{subsec:3.a}}

Al multiplicar potencias de igual base, los exponentes se suman. Y al dividir potencias de igual base, los exponentes se restan.

\begin{itemize}
\item $ a^b \cdot a^c = a^{b + c} \quad \forall a, b, c \in \mathbb{R}$
\item $ \frac{a^b}{a^c} = a^{b - c} \quad \forall a, b, c \in \mathbb{R}$
\end{itemize}

\begin{equation}
\frac{3^4 \cdot 3^7}{3^{12}} = \frac{3^{11}}{3^{12}} = 3^{-1} = \tcboxmath[colback=orange!25!white,colframe=orange, title=3.a] { \hspace{0.25em} \frac{1}{3} \hspace{0.25em} }
\end{equation}

\subsection*{3.b}
\label{subsec:3.b}
\addcontentsline{toc}{subsection}{\nameref{subsec:3.b}}

En el ejercicio 2.a, se vio cómo lidiar con exponentes de la forma $\frac{1}{n}$, con $n$ entero. Para el caso más general de un exponente fraccionario, puede descomponerse la exponenciación en un paso entero y un paso de la forma $\frac{1}{n}$. Verbigracia:

\begin{equation}
a^{\frac{p}{q}} = a^{(p \frac{1}{q})} = (a^p)^{\frac{1}{q}}
\end{equation}

Así, si resulta conveniente, es posible evaluar $a^p$ primero y luego tomar la q-ésima raíz de ese valor.

Aplicando todo lo ya visto a este caso:

\begin{subequations}
\begin{align}
& \sqrt{ \frac{ (5 \cdot 10^{-6}) (4 \cdot {10}^2) }{8 \cdot {10}^5} } = \\
& \sqrt{ \frac{ 20 \cdot 10^{-4} }{8 \cdot {10}^5} } = \\
& \sqrt{ \frac{5}{2} \cdot 10^{-9} } = \\
& \sqrt{ \frac{5}{2} } \cdot 10^{-\frac{9}{2}} = \\
& \frac{\sqrt{5}}{\sqrt{2}} \cdot \frac{\sqrt{2}}{\sqrt{2}} \cdot 10^{-\frac{9}{2}} = \\
& \frac{\sqrt{10}}{2} \cdot 10^{-4} \cdot 10^{-\frac{1}{2}} = \\
& \frac{1}{2} \cdot \cancel{10^{\frac{1}{2}}} \cdot 10^{-4} \cdot \cancel{10^{-\frac{1}{2}}} = \\
& \frac{1}{2} \cdot 10 \cdot 10^{-5} = \tcboxmath[colback=orange!25!white,colframe=orange, title=3.b] { 5 \cdot 10^{-5} }
\end{align}
\end{subequations}

\subsection*{3.c}
\label{subsec:3.c}
\addcontentsline{toc}{subsection}{\nameref{subsec:3.c}}

Al llegar al final de éste, conviene sumar todas las fracciones por un lado y los enteros por otro, para minimizar las chances de error. Teniendo mucho cuidado, debería dar el siguiente valor:

\begin{equation}
\tcboxmath[colback=orange!25!white,colframe=orange, title=3.c] { \frac{8039}{72} = 111.652\overline{7} }
\end{equation}

\subsection*{3.d}
\label{subsec:3.d}
\addcontentsline{toc}{subsection}{\nameref{subsec:3.d}}

La trampa en este ejercicio consiste en los signos negativos en los radicales. No es válido cancelar un exponente interno con uno externo cuando hay un signo negativo involucrado. Por ejemplo:

\begin{equation}
\sqrt{(-3)^2} = \sqrt{9} = 3
\end{equation}

Una resolución incorrecta cancelando exponentes sería:

\begin{equation}
\sqrt{(-3)^2} = {(-3)^2}^{\frac{1}{2}} = -3 \quad \textcolor{red}{\text{INCORRECTO}}
\end{equation}

La moraleja es que si el radicando tiene signo negativo, es preferible evaluar la potencia en dos pasos, considerando el signo. Cancelar exponentes puede conducir a errores. Con estas precauciones en mente, el resultado debería ser:

\begin{equation}
\tcboxmath[colback=orange!25!white,colframe=orange, title=3.d] { \frac{31}{3} + 3 \sqrt[3]{3} \approx 14,660 }
\end{equation}

\section*{4}
\label{sec:4}
\addcontentsline{toc}{section}{\nameref{sec:4}}

\textbf{Si } $x = -2; y = \frac{2}{3}; z = -\frac{3}{2}$, \textbf{calcule:}

\begin{enumerate}[(a)]
\bfseries

\item $ x \cdot (y + z) $

\item $ x \cdot y + z $

\item $ x + y \cdot z $

\item $ (x + y) \cdot z $

\end{enumerate}
\hrule

\vspace{1em}

Sólo hay que reemplazar los valores y realizar las cuentas, que son triviales.

\begin{subequations}
\begin{align}
& x \cdot (y + z) = \tcboxmath[colback=orange!25!white,colframe=orange, title=4.a] { \hspace{0.25em} \frac{5}{3} \hspace{0.25em} } \\
& x \cdot y + z = \tcboxmath[colback=orange!25!white,colframe=orange, title=4.b] { \hspace{0.5em} -\frac{17}{6} \hspace{0.5em} } \\
& x + y \cdot z = \tcboxmath[colback=orange!25!white,colframe=orange, title=4.c] { \hspace{0.25em} -3 \hspace{0.25em} } \\
& (x + y) \cdot z = \tcboxmath[colback=orange!25!white,colframe=orange, title=4.d] { \hspace{0.5em} 2 \hspace{0.5em} }
\end{align}
\end{subequations}

\end{document}
