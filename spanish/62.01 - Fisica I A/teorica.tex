\documentclass{article}

\usepackage{amsmath}
\usepackage{amssymb}
\usepackage{enumerate}
\usepackage[spanish]{babel}
\usepackage{cancel}
\usepackage{caption}
\usepackage[margin=1.5in]{geometry}
\usepackage{graphicx}
\usepackage[utf8]{inputenc}
\usepackage{tcolorbox}
\usepackage{esint}
\usepackage{hyperref}
\hypersetup{
    colorlinks,
    citecolor=black,
    filecolor=black,
    linkcolor=black,
    urlcolor=black,
}

\renewcommand{\Bbb}{\mathbb}

\tcbuselibrary{theorems}

\title{Apuntes teórico-prácticos de Física I A (62.01) \\ Cátedra Menikheim \\ 1° C 2004}
\author{Darío Eduardo Ramos}

\definecolor{ududff}{rgb}{0.30196078431372547,0.30196078431372547,1}
\definecolor{cqcqcq}{rgb}{0.7529411764705882,0.7529411764705882,0.7529411764705882}

\begin{document}
\maketitle

\tableofcontents{}
\newpage

\section{Cinemática}

\subsection{Definiciones iniciales}

La \textbf{cinemática} es la rama de la física que describe el movimiento de los objetos sólidos sin considerar las causas que lo originan. 

El primer \textbf{modelo} que se adoptará para los cuerpos físicos a modelar será el de \textbf{punto material o partícula}: dichos cuerpos físicos serán considerados como sin volumen, pero sí con masa. Toda esa masa estará ``concentrada'' en el centro de masa del cuerpo, que será el punto que represente al objeto.

\textbf{Movimiento}: Un cuerpo será considerado en movimiento cuando cambie de posición a lo largo del tiempo respecto a un sistema de referencia considerado fijo o inercial. Más sobre esto último cuando se estudien los sistemas inerciales y no inerciales.

Cuando el cuerpo se desplaza desde una posición inicial dada por el vector $\overrightarrow{ r_i }$ a una posición final dada por el vector $\overrightarrow{ r_f }$, en un tiempo $\Delta t$, se definen las siguientes magnitudes vectoriales y escalares:

\begin{itemize}
\item \textbf{Vector desplazamiento:} $\overrightarrow{ \Delta r } = \overrightarrow{r_f} - \overrightarrow{r_i}$. Por como está definido, el vector $\overrightarrow{\Delta r}$ va de $\overrightarrow{r_i}$ hacia $\overrightarrow{r_f}$.
\item \textbf{Camino recorrido (escalar):} longitud del arco de trayectoria entre posición inicial y final.
\item \textbf{Vector velocidad media:} $\overrightarrow{v_m} = \frac{\overrightarrow{\Delta r}}{\Delta t}$
\item \textbf{Rapidez media (escalar):} $\overline{v} = \frac{\Delta s}{\Delta t}$
\item \textbf{Vector velocidad instantánea:}

\begin{equation}
\overrightarrow{v} = \lim_{\Delta t \rightarrow 0} \frac{ \mathop{\overrightarrow{\Delta r}} }{\Delta t} = \frac{d\overrightarrow{r}}{\mathop{dt}}
\end{equation}

Nótese entonces que la velocidad instantánea es la derivada de la posición respecto al tiempo. Además, al ser $\overrightarrow{r}(t)$ una función vectorial del tiempo, $\overrightarrow{v}$ también lo es. Punto a punto, el vector $\overrightarrow{v}(t_0)$ es tangente a la curva de la trayectoria en el punto $\overrightarrow{r}(t_0)$ y su sentido es el del movimiento.
\item \textbf{Rapidez instantánea:}

\begin{equation}
v = \lim_{\Delta t \rightarrow 0} \frac{\Delta s}{\Delta t} = \frac{\mathop{ds}}{\mathop{dt}}
\end{equation}

\item \textbf{Vectores aceleración media e instantánea:}

\begin{equation}
\overrightarrow{a_m} = \frac{ \overrightarrow{ \Delta v } }{ \Delta t }
\end{equation}

\begin{equation}
\overrightarrow{a} = \lim_{\Delta t \rightarrow 0} \frac{ \overrightarrow{ \Delta v } }{\Delta t} = \frac{ \mathop{d\overrightarrow{v}} }{ \mathop{dt} } = \frac{ \mathop{ d^2 \overrightarrow{r} } }{ \mathop{ dt^2 } }
\end{equation}

Cualquier cambio en el sentido, dirección o norma del vector velocidad indica que existe aceleración. Esto implica que en todo movimiento curvo hay aceleración no nula.

\end{itemize}

\end{document}