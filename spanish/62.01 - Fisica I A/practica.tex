\documentclass{article}

\usepackage{hyperref}
\hypersetup{
    colorlinks,
    citecolor=black,
    filecolor=black,
    linkcolor=black,
    urlcolor=black,
}

\title{Apuntes prácticos de Física I A (62.01) \\ Cátedra Menikheim \\ 1° C 2004}
\author{Darío Eduardo Ramos}

\begin{document}
\maketitle
\tableofcontents{}
\newpage

\section{Requisitos para informes de trabajos prácticos}

\begin{enumerate}
\item \textbf{Carátula} con título del TP, materia y número de curso, profesor, número de grupo, integrantes, fecha de realización, fecha de entrega y de re-entregas, si hay.
\item \textbf{Resumen:} Alrededor de 8 renglones que resuman lo hecho y los resultados obtenidos. Debe ser conciso y comprensible para cualquier lector.
\item \textbf{Objetivos:} Concretos y puntuales. Generales. Deben estar narrados, no esquematizados.
\item \textbf{Descripción de los elementos utilizados:} Listado de los mismos.
\item \textbf{Introducción teórica:} Conceptos necesarios para el TP.
\item \textbf{Desarrollo/procedimiento:} Lo que se hizo, narrado con detalle. Criterios empleados, esquema(s) del(de los) dispositivo(s) empleado(s), fórmulas aplicadas.
\item \textbf{Resultados y discusión:} Además de presentar los resultados en sí, debe haber un análisis profundo de los mismos.
\item \textbf{Conclusiones:} Sobre el TP, sobre el tema analizado, etc.
\item \textbf{Apéndices:} Información concreta que, por su especificidad o complejidad, entorpecerían la lectura de estar en el cuerpo del informe.
\item \textbf{Bibliografía.}
\end{enumerate}

La hoja con las mediciones debe estar firmada por los docentes y entregada junto al informe.

\end{document}