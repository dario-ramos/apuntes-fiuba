\documentclass{article}

\usepackage[spanish]{babel}
\usepackage[utf8]{inputenc}

\begin{document}

\section{Ejercicio 1}

Para la señal $x(t) = \frac{1}{\sqrt{2 \pi} \sigma} e^{\frac{-1}{2} \frac{(t-\mu)^2}{\sigma^2}}$ (gaussiana),

(a) Calcular aproximadamente la frecuencia de Nyquist de muestreo observando con la FFT el
espectro de la señal obtenida. Contrastar el resultado obtenido teóricamente. (Para el cálculo
teórico considere que una gaussiana se hace cero a más o menos $3 \sigma$ de la media).

(b) Repetir el punto anterior pero para la señal x(t) multiplicada por y(t) = cos(2 f 0 t).

Para tener una mejor idea del problema, primero se resolverá de forma teórica/analítica.

%\begin{equation}
%X(\omega) = \int_{-\infty}^{+\infty} x(t) e^{-j \omega t}
%\end{equation}

\[
X(\omega) = \int_{-\infty}^{+\infty} x(t) e^{-j \omega t}
= \int_{-\infty}^{+\infty} frac 1{sqrt(2 \pi) \sigma} e
\]

\end{document}