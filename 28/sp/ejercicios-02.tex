\documentclass{article}

\usepackage{amsmath}
\usepackage{amssymb}
\usepackage[spanish]{babel}
\usepackage{enumerate}
\usepackage{hyperref}
\hypersetup{
    colorlinks,
    citecolor=black,
    filecolor=black,
    linkcolor=black,
    urlcolor=black,
}
\usepackage{tcolorbox}

\tcbuselibrary{theorems}

\newcommand{\hresult}[2]{\tcboxmath[colback=orange!25!white,colframe=orange, title=#1] {#2} }
\newcommand{\figurex}[4]{\begin{figure}[ht] \caption{#1} \includegraphics[scale=#2]{#3} \centering \label{#4}\end{figure}}
\newcommand{\sectionx}[1]{\section*{#1}\label{sec:#1}\addcontentsline{toc}{section}{\nameref{sec:#1}}}
\newcommand{\subsectionx}[1]{\subsection*{#1}\label{subsec:#1}\addcontentsline{toc}{subsection}{\nameref{subsec:#1}}}

\renewcommand{\Bbb}{\mathbb}

\title{Ejercicios de Análisis Matemático CBC (28) \\
Práctica 2: La Recta real \\
Cátedra Ruiz, curso 62802 \\
1° C 2003}
\author{Darío Eduardo Ramos}

\begin{document}
\maketitle

\tableofcontents{}

\newpage

\sectionx{1}

\textbf{Representar en la recta numérica:}

\begin{enumerate}[(a)]

\bfseries

\item $ -5; -1; 3; 6; \frac{3}{8}; 1+\frac{2}{5}; 1-\frac{2}{5}; -\sqrt{2}; \sqrt{2}+1; \sqrt{2}-1; -\sqrt{2}+1; -\sqrt{2}-1 $

\item $ -3; -2; -1; 0; 1; 2; 3; -\pi; -\frac{\pi}{2}; \frac{\pi}{2}; \pi; \frac{3}{2}\pi; 3,14; -3,14 $

\end{enumerate}

\hrule
\vspace{1em}

Graficar números reales en la recta equivale a ordenarlos. Con saber los valores de cada uno, aproximando si hace falta, es suficiente. Para las fracciones, siempre se puede obtener un valor exacto, salvo que sea un decimal periódico. Para las expresiones irracionales, habrá que tomar una cantidad suficiente de decimales que permita la comparación. Con todo esto en mente, el resultado será presentado como el orden final de los números, ya que el gráfico es trivial.

\subsectionx{1.a}

Valores de menor a mayor:

\begin{enumerate}[(I)]

\item $ -5 $

\item $ -\sqrt{2}-1 \approx -2,41 \dots $

\item $ -\sqrt{2} \approx -1,41 \dots $

\item $ -1 $

\item $ -\sqrt{2}+1 \approx -0,41 \dots $

\item $ \frac{3}{8} = 0,375 $

\item $ \sqrt{2}-1 \approx 0,41 \dots $

\item $ 1-\frac{2}{5} = \frac{3}{5} = 0,6 $

\item $ 1+\frac{2}{5} = \frac{7}{5} = 1,4 $

\item $ \sqrt{2}+1 \approx 2,41 \dots $

\item $ 3 $

\item $ 6 $

\end{enumerate}

\subsectionx{1.b}

Valores de menor a mayor:

\begin{enumerate}[(I)]

\item $ -\pi \approx -3,1415 \dots $

\item $ -3,14 $

\item $ -3 $

\item $ -2 $

\item $ -\frac{\pi}{2} \approx -1,5707 \dots $

\item $ -1 $

\item $ 0 $

\item $ 1 $

\item $ \frac{\pi}{2} \approx 1,5707 \dots $

\item $ 2 $

\item $ 3 $

\item $ 3,14 $

\item $ \pi \approx 3,1415 \dots $

\end{enumerate}


\sectionx{2}

\textbf{Represente en la recta numérica los siguientes conjuntos. Escribirlos como intervalos o unión de intervalos.}


\begin{enumerate}[(a)]

\bfseries

\item Todos los números reales mayores que -1.

\item Todos los números reales menores o iguales que 2.

\item Todos los números reales que distan del 0 menos que 3.

\item $ \{ x \in \mathbb{R} / 2x - 3 > 5 \} $

\item $ \{ x \in \mathbb{R} / -3 < x \leq 3 \} $

\item $ \{ x \in \mathbb{R} / 1 < 2x-3 < 5 \} $

\item $ \{ x \in \mathbb{R} / x(2x-3) > 0 \} $

\item $ \{ x \in \mathbb{R} / x^2 - 36 < 0 \} $

\item $ \{ x \in \mathbb{R} / x^3 - x < 0 \} $

\item $ \{ x \in \mathbb{R} / 1 + \frac{2}{x} < 3 \} $

\item $ \{ x \in \mathbb{R} / \frac{1}{x} < \frac{4}{x} \} $

\item $ \{ x \in \mathbb{R} / |x| < 3 \} $

\item $ \{ x \in \mathbb{R} / |x-2| < 3 \} $

\item $ \{ x \in \mathbb{R} / |x+2| < 3 \} $

\item $ \{ x \in \mathbb{R} / |x| > 3 \} $

\end{enumerate}
\hrule
\vspace{1em}

Representar gráficamente un conjunto en la recta real no tiene mucho más misterio que colorear un intervalo o unión de intervalos. Para representar que un extremo no se incluye por desigualdad estricta, se suele dibujar un círculo vacío alrededor del punto a excluir. Cuando la desigualdad no es estricta, se pinta el interior del círculo. Por brevedad, se harán algunos gráficos como ejemplos y se omitirá el resto.

\subsectionx{2.a}

Si se denomina $A$ al conjunto pedido, expresarlo como intervalo es directo:

\begin{equation}
\hresult{2.a}{ A = (-1, +\infty) }
\end{equation}

\figurex{Ejercicio 2.a}{4.5}{../img/guide_02/ex_02a.png}{fig:2a}

\subsectionx{2.b}

\begin{equation}
\hresult{2.b}{ A = (-\infty, 2] }
\end{equation}

\figurex{Ejercicio 2.b}{4.5}{../img/guide_02/ex_02b.png}{fig:2b}

\subsectionx{2.c}

\begin{equation}
\hresult{2.c}{ A = (-3, 3) }
\end{equation}

\subsectionx{2.d}

En este caso hay que transformar la desigualdad un poco:

\begin{subequations}
\begin{align}
& 2x-3 > 5 \\
& 2x > 8 \\
& x > 4
\end{align}
\end{subequations}

\begin{equation}
\hresult{2.d}{ A = (4, +\infty) }
\end{equation}

\subsectionx{2.e}

\begin{equation}
\hresult{2.e}{ A = (-3, 3] }
\end{equation}

\subsectionx{2.f}

\begin{subequations}
\begin{align}
& 1 < 2x-3 < 5 \\
& 4 < 2x < 8 \\
& 2 < x < 4
\end{align}
\end{subequations}

\begin{equation}
\hresult{2.f}{ A = (2, 4) }
\end{equation}

\subsectionx{2.g}

En este caso se tiene un producto, por lo que hay que considerar dos casos usando la regla de los signos.

En primer lugar, que ambos factores sean positivos y no nulos:

\begin{equation}
x > 0 \wedge 2x-3 > 0 \Rightarrow x > 0 \wedge x > \frac{3}{2} \Rightarrow x \in \left( \frac{3}{2}, +\infty \right)
\end{equation}

En segundo y último lugar, que ambos factores sean negativos a la vez.

\begin{equation}
x < 0 \wedge 2x-3 < 0 \Rightarrow x < 0 \wedge x < \frac{3}{2} \Rightarrow x \in (-\infty, 0)
\end{equation}

El resultado final es la unión de los intervalos asociados a cada caso:

\begin{equation}
\hresult{2.g}{ A = (-\infty, 0) \cup \left( \frac{3}{2}, +\infty \right) }
\end{equation}

\subsectionx{2.h}

Recuérdese que cuando se eleva un valor al cuadrado y luego se toma la raíz cuadrada, no se puede cancelar raíz con cuadrado sin considerar el módulo. Esto es porque al elevar los valores negativos al cuadrado, el resultado es positivo y puede tomarse la raíz.

\begin{subequations}
\begin{align}
& x^2 - 36 < 0 \\
& x^2 < 36 \\
& \sqrt{x^2} < 6 \\
& |x| < 6 \\
& -6 < x < 6
\end{align}
\end{subequations}

\begin{equation}
\hresult{2.h}{ A = (-6, 6) }
\end{equation}

\subsectionx{2.i}

Siempre que se analiza el signo, conviene llevar la expresión a un producto de dos factores.

\begin{subequations}
\begin{align}
& x^3 - x < 0 \\
& x (x^2-1) < 0
\end{align}
\end{subequations}

Caso 1: primer factor positivo, segundo negativo.

\begin{equation}
x > 0 \wedge x^2-1 < 0 \Rightarrow x > 0 \wedge |x| < 1  \Rightarrow x \in (0, 1)
\end{equation}

Caso 2: primer factor negativo, segundo positivo.

\begin{equation}
x < 0 \wedge x^2-1 > 0 \Rightarrow x < 0 \wedge |x| > 1  \Rightarrow x \in (-\infty, -1)
\end{equation}

\begin{equation}
\hresult{2.i}{ A = (-\infty, -1) \cup (0,1) }
\end{equation}

\subsectionx{2.j}

\begin{subequations}
\begin{align}
& 1 + \frac{2}{x} < 3 \\
& \frac{2}{x} < 2 \\
& \frac{1}{x} < 1
\end{align}
\end{subequations}

Llegado este punto, no se puede multiplicar miembro a miembro por $ x $ directamente, porque si $ x $ es negativo hay que invertir la desigualdad. Por lo tanto, se consideran ambos casos:

Caso 1: x positivo.

\begin{equation}
x > 0 \wedge 1 < x \Rightarrow x \in (1, +\infty)
\end{equation}

Caso 2: x negativo.

\begin{equation}
x < 0 \wedge 1 > x \Rightarrow x \in (-\infty, 0)
\end{equation}

\begin{equation}
\hresult{2.j}{ A = (-\infty, 0) \cup (1, +\infty) }
\end{equation}

\subsectionx{2.k}

Nuevamente, hay que considerar el signo de $ x $.

Caso 1: x positivo.

\begin{equation}
x > 0 \wedge \frac{1}{x} < \frac{4}{x} \Rightarrow x > 0 \wedge 1 < 4 \Rightarrow x \in (0, +\infty)
\end{equation}

Caso 2: x negativo.

\begin{equation}
x < 0 \wedge \frac{1}{x} < \frac{4}{x} \Rightarrow x < 0 \wedge 1 > 4 \Rightarrow x \in \emptyset
\end{equation}

Como el caso 2 no aporta elementos, resulta:

\begin{equation}
\hresult{2.k}{ A = (0, +\infty) }
\end{equation}

\subsectionx{2.l}

De manera general, $ |x| < k \Rightarrow -k < x < k $. Por lo tanto:

\begin{equation}
\hresult{2.l}{ A = (-3, 3) }
\end{equation}

\subsectionx{2.m}

Hay que analizar el signo del argumento del módulo.

Caso 1: $ x-2 \geq 0 \Rightarrow x - 2 < 3 \Rightarrow x \geq 2 \wedge x < 5 \Rightarrow x \in [2, 5) $.

Caso 2: $ x - 2 < 0 \Rightarrow -x+2 < 3 \Rightarrow x < 2 \wedge x > -1 \Rightarrow x \in (-1, 2)  $

Combinando los dos intervalos, resulta entonces:

\begin{equation}
\hresult{2.m}{ A = (-1, 5) }
\end{equation}

\subsectionx{2.n}

Caso 1: $ x+2 \geq 0 \Rightarrow x + 2 < 3 \Rightarrow x \geq -2 \wedge x < 1 \Rightarrow x \in [-2, 1) $.

Caso 2: $ x+2 < 0 \Rightarrow -x-2 < 3 \Rightarrow x < -2 \wedge x > -5 \Rightarrow x \in (-5, -2)  $

\begin{equation}
\hresult{2.n}{ A = (-5, 1) }
\end{equation}

\subsectionx{2.ñ}

De manera general, para $ k \in \mathbb{R} \wedge k > 0 $, $ |x| > k \Rightarrow x > k \wedge x < -k \Rightarrow x \in \{ (-\infty, -k) \cup (k, +\infty) \} $,.

\begin{equation}
\hresult{2.ñ}{ A = (-\infty, -3) \cup (3, +\infty) }
\end{equation}

\sectionx{3}

\textbf{Representar en la recta los siguientes conjuntos:}

\begin{enumerate}[(a)]

\bfseries

\item $ [2, 4] \cap [3, 6] $

\item $ [2, 4] \cup [3, 6] $

\item $ (-\infty, 3) \cap (1, +\infty) $

\item $ (-1, 3) \cap [3, +\infty) $

\item $ (-1, 3) \cup [3, +\infty) $

\item $ (-1, 3) \cup (3, 5) $

\end{enumerate}

\hrule
\vspace{1em}

Nuevamente, dibujar conjuntos en la recta es trivial, por lo que sólo se expresarán los resultados como intervalos o unión de intervalos.

\subsectionx{3.a}

Cuando dos intervalos tienen una intersección no nula, como en este caso, la misma puede obtenerse tomando como límite inferior el máximo de los límites inferiores y como límite superior el mínimo de dichos límites. Para este caso, resulta:

\begin{equation}
\hresult{3.a}{ A = [3, 4] }
\end{equation}

\subsectionx{3.b}

Al considerar la unión de intervalos no disjuntos, se aplica un criterio similar pero inverso: el límite inferior de la unión es el mínimo de los límites inferiores, y el límite superior es el máximo. Para este caso, se tiene:

\begin{equation}
\hresult{3.b}{ A = [2, 6] }
\end{equation}

\subsectionx{3.c}

\begin{equation}
\hresult{3.c}{ A = (1, 3) }
\end{equation}

\subsectionx{3.d}

Estos intervalos no tienen ni un solo punto en común. Uno incluye el 3, pero el otro no.

\begin{equation}
\hresult{3.d}{ A = \emptyset }
\end{equation}

\subsectionx{3.e}

Al incluir el punto 3, resulta un solo intervalo.

\begin{equation}
\hresult{3.e}{ A = (-1, +\infty) }
\end{equation}

\subsectionx{3.f}

En este caso, 3 no pertenece a ninguno de los intervalos, por lo que no hay cambio.

\begin{equation}
\hresult{3.f}{ A = (-1, 3) \cup (3, 5) }
\end{equation}

\sectionx{4}

\textbf{Representar en la recta los siguientes conjuntos:}

\begin{enumerate}[(a)]

\bfseries

\item $ \{ n \in \mathbb{N} / 4 \leq n < 6 \} $

\item $ \{ n \in \mathbb{N} / n < 13 \} $

\item $ \left\{ x = \frac{n}{n+1} / n \in \mathbb{N} \wedge n < 6 \right\} $

\item $ \left\{ x = \frac{n}{n+1} / n \in \mathbb{N} \right\} $

\end{enumerate}

\hrule
\vspace{1em}

En este caso, es importante tener en cuenta que si se están representando números naturales, el conjunto serán puntos y no segmentos de recta como antes.

\subsectionx{4.a}

Dado que los números han sido restringidos a naturales, resulta:

\begin{equation}
\hresult{4.a}{ A = \{ 4, 5 \} }
\end{equation}

El gráfico de esto consistiría simplemente de los puntos 4 y 5 sobre el eje $ x $.

\subsectionx{4.b}

Asumiendo la convención de que cero no es un número natural, resulta:

\begin{equation}
\hresult{4.b}{ A = \{ 1, 2, 3, \dots, 10, 11, 12 \} }
\end{equation}

\subsectionx{4.c}

Nótese que los valores de $ x $ son número racionales que dependen de $ n $. Dado que $ n $ está acotado, resulta:

\begin{equation}
\hresult{4.b}{ A = \left\{ \frac{1}{2}, \frac{2}{3}, \frac{3}{4}, \frac{4}{5}, \frac{5}{6} \right\} }
\end{equation}

\subsectionx{4.d}

Es lo mismo que el inciso anterior pero hasta el infinito. Vale decir:

\begin{equation}
\hresult{4.b}{ A = \left\{ \frac{1}{2}, \frac{2}{3}, \frac{3}{4}, \frac{4}{5}, \frac{5}{6}, \frac{6}{7}, \frac{7}{8}, \dots \right\} }
\end{equation}

Gráficamente, esto es una acumulación de puntos que se aproximan infinitesimalmente a 1.

\sectionx{5}

\textbf{ Demostrar que $ \sqrt{3} $ no es racional. }

\vspace{1em}
\hrule
\vspace{1em}

La forma más directa de demostrar esto es vía el método de reducción al absurdo. Si $ \sqrt{3} $ es racional, por definición existen $ p $ y $ q $ enteros y sin factores en común aparte de 1 tales que:

\begin{equation}
\sqrt{3} = \frac{p}{q}, p \in \mathbb{Z}, q \in \mathbb{Z}, \text{ con p y q coprimos }
\end{equation}

Recuérdese que exigir que dos enteros no tengan factores en común además de 1 es la definición de números coprimos.

Elevando ambos lados al cuadrado:

\begin{equation}
3 = \frac{p^2}{q^2}
\end{equation}

Multiplicando miembro a miembro por $ q^2 $:

\begin{equation}
3 q^2 = p^2
\end{equation}

Esta igualdad indica que $ p^2 $ es divisible por 3, ya que puede expresarse como $ 3 q^2 $. En este punto, se utiliza el siguiente resultado:

\begin{equation}
a \in \mathbb{Z} \wedge k \text{ primo } \wedge k | a^2 \Rightarrow k | a
\end{equation}

Vale decir, si $ a^2 $ es divisible por un número primo $ k $, entonces $ a $ también. Aplicando esto a este caso, al ser 3 un número primo y $ p^2 $ divisible por 3, resulta entonces que $ p $ es divisible por 3.

Ahora bien, si $ p $ es divisible por 3, existe un entero $ \alpha $ tal que $ p = 3 \alpha $. Reemplazando:

\begin{subequations}
\begin{align}
& 3 q^2 = (3 \alpha)^2 \\
& 3 q^2 = 9 \alpha^2 \\
& q^2 = 3 \alpha^2
\end{align}
\end{subequations}

Esto implica que $ q^2 $ es divisible por 3. Utilizando el mismo resultado de antes, eso implica que $ q $ también es divisible por 3.

Es en este punto donde aparece la contradicción: la hipótesis inicial de que $ p $ y $ q $ son coprimos se contradice con el hecho de que ambos sean divisibles por 3. Por lo tanto, $ \sqrt{3} $ es irracional.

\sectionx{6}

\textbf{ Dados los números $ 3,14 $ y $ \pi $: }

\begin{enumerate}[(a)]

\bfseries

\item Hallar un número racional comprendido entre ambos.

\item Hallar un número irracional comprendido entre ambos (ayuda: escribir su desarrollo decimal).

\end{enumerate}

\hrule

\subsectionx{ 6.a }

Dado que $ \pi \approx 3,14159 \dots $, un número racional entre $ 3,14 $ y $ \pi $ es $ 3,141 $. Pasado a fracción:

\begin{equation}
\hresult{ 6.a }{ 3,141 = \frac{3141}{1000} }
\end{equation}

\subsectionx{ 6.b }

Una propiedad útil para este caso es la siguiente: si a un número irracional se le suma un número racional, se obtiene un número irracional. Lo mismo ocurre si se lo multiplica por un número racional. Por lo tanto, si se toma el promedio entre $ 3,14 $ y $ \pi $, se obtendrá un número entre ambos. Nótese que esto puede repetirse \textit{ad infinitum}, por ende hay infinitos números irracionales entre $ \pi $ y $ 3,14 $.

\begin{equation}
\hresult{6.b}{ \frac{3,14 + \pi}{2} \approx 3.1407963 \dots }
\end{equation} 

Esto puede generalizarse y por eso se dice que los irracionales son un conjunto denso dentro de los reales.

\sectionx{7}

\textbf{Considerar los siguientes conjuntos:}

\begin{enumerate}[(A)]

\bfseries

\item $ A = \left\{ \frac{1}{n} : n \in \mathbb{N} \right\} $

\item $ B = \left\{ \frac{n}{n+1} : n \in \mathbb{N} \right\} $

\item $ C = (0, 7) $

\item $ D = \mathbb{N} $

\item $ E = \left\{ n \frac{1}{n^2} : n \in \mathbb{N} \right\} $

\item $ F = \{ 1, 2, 3, 4 \} $

\item $ G = \{ 5; 5,9; 5,99; 5,999; \dots \} $

\item $ H = \{ x \in \mathbb{R} / |x-2| < 1 \} $

\item $ I = \{ x \in \mathbb{R} / |x| > 3 \} $

\end{enumerate}

En cada caso:

\begin{enumerate}[(a)]

\bfseries

\item Determinar si 7 es una cota superior.

\item Determinar si 0 es una cota inferior.

\item Decidir si está acotado superiormente.

\item Decidir si está acotado inferiormente.

\item En caso afirmativo, encontrar el supremo y/o el ínfimo del conjunto. Decidir si alguno de ellos es el máximo o el mínimo del conjunto correspondiente.

\end{enumerate}

\hrule
\vspace{1em}

Antes de comenzar, se repasarán las definiciones. Dados un conjunto $ A \subset \mathbb{R} $ y un número real $ M $:

\begin{itemize}

\item \textbf{Máximo:} $ A $ tiene elemento máximo $ \Leftrightarrow \exists a_0 \in A / \forall a \in A, a \le a_0 $

\item \textbf{Mínimo:} $ A $ tiene elemento mínimo $ \Leftrightarrow \exists b_0 \in A / \forall b \in A, b \ge b_0 $

\item \textbf{Cota superior:} $ M $ es cota superior de A $ \Leftrightarrow \forall a \in A, a \leq M $

\item \textbf{Cota inferior:} $ M \text{ es cota inferior de A } \Leftrightarrow \forall a \in A, a \geq M $

\item \textbf{Conjunto acotado superiormente:} A está acotado superiormente si y sólo si tiene al menos una cota superior. De hecho, o no tiene, o tiene infinitas, porque si M es cota superior, también lo son M+1, M+2, ...

\item \textbf{Conjunto acotado inferiormente:} A está acotado inferiormente si y sólo si tiene una cota inferior. Al igual que con las superiores, o no tiene ninguna, o tiene infinitas.

\item \textbf{Conjunto acotado:} A está acotado, a secas, si y sólo si está acotado superior e inferiormente a la vez.

\item \textbf{Supremo:} Dado A acotado superiormente, su supremo es la mínima cota superior.

\item \textbf{Ínfimo:} Dado A acotado inferiormente, su ínfimo es la máxima cota inferior.

\end{itemize}

Nótese que si existen, el máximo, mínimo, supremo e ínfimo son únicos para cada conjunto. En cambio, las cotas, si existen, siempre son infinitas.

\subsectionx{7.A}

\begin{equation}
A = \left\{ \frac{1}{n} : n \in \mathbb{N} \right\}
\end{equation}

Todo elemento de $ A $ es una fracción. El numerador siempre es 1, y el denominador comienza en 1 y va creciendo. Por lo tanto, el máximo valor de A es el asociado a $ n = 1 $ y vale 1. Al crecer $ n $, crece el denominador y la fracción tiende a cero pero nunca lo alcanza. Por ende, $ A $ está acotado:

\begin{equation}
\forall a \in A, 0 < a \le 1
\end{equation}

\begin{enumerate}[(a)]

\item ¿7 es una cota superior? Sí, porque todo elemento de A es menor que 7.

\item ¿0 es una cota inferior? Sí, porque todo elemento de A es mayor que 0.

\item ¿A está acotado superiormente? Sí, porque todo elemento es menor o igual a 1.

\item ¿A está acotado inferiormente? Sí, porque todo elemento es mayor que 0.

\item Supremo, ínfimo, mínimo y máximo: en base al análisis inicial:

\begin{itemize}

\item $ \mathop{sup}(A) = 1 $

\item $ \mathop{max}(A) = 1 $

\item $ \mathop{inf}(A) = 0 $

\item $ \mathop{min(A)} = \nexists $
 
\end{itemize}

El mínimo no existe porque siempre se puede encontrar un elemento de A más cercano al cero que el anterior. Y el valor cero no pertenece al conjunto, por ende hay ínfimo pero no hay mínimo en este caso.

\end{enumerate}

\subsectionx{7.B}

\begin{equation}
B = \left\{ \frac{n}{n+1} : n \in \mathbb{N} \right\}
\end{equation}

Analizando los primeros valores:

\begin{equation}
A = \left\{ \frac{1}{2}, \frac{2}{3}, \frac{3}{4}, \dots \right\}
\end{equation}

Cuando $ n $ tiende a infinito, el 1 en el numerador se vuelve despreciable comparado a $ n $, por lo tanto el límite es 1. A simple vista, los valores comienzan en $ \frac{1}{2} $ y crecen monótonamente acercándose a 1. Esto puede comprobarse formalmente analizando el cociente $ \frac{a_{n+1}}{a_n} $:

\begin{equation}
\frac{a_{n+1}}{a_n} = \frac{n+1}{n+2} : \frac{n}{n+1} = \frac{(n+1)^2}{n(n+2)} = \frac{n^2 + 2n + 1}{n^2 + 2n}
\end{equation}

Nótese que el numerador es el denominador +1, por ende, para todo n:

\begin{equation}
\frac{a_{n+1}}{a_n} > 1 \Rightarrow a_{n+1} > a_{n} \forall n
\end{equation}

Tener en cuenta que multiplicar miembro a miembro por $ a_n $ es válido en este caso particular porque $ a_n $ es positivo para todo n.

\begin{enumerate}[(a)]

\item ¿7 es una cota superior? Sí, porque todo elemento de A es menor que 7.

\item ¿0 es una cota inferior? Sí, porque todo elemento de A es mayor que 0.

\item ¿A está acotado superiormente? Sí, porque todo elemento es menor a 1.

\item ¿A está acotado inferiormente? Sí, porque todo elemento es mayor o igual que $ \frac{1}{2} $.

\item Supremo, ínfimo, mínimo y máximo: en base al análisis inicial:

\begin{itemize}

\item $ \mathop{sup}(A) = 1 $

\item $ \mathop{max}(A) = \nexists $

\item $ \mathop{inf}(A) = \frac{1}{2} $

\item $ \mathop{min(A)} = \frac{1}{2} $
 
\end{itemize}

El máximo no existe porque siempre se puede encontrar un elemento de A más cercano a 1 que el anterior. Y el valor 1 no pertenece al conjunto, por ende hay supremo pero no hay máximo en este caso.

\end{enumerate}

\subsectionx{7.C}

\begin{equation}
C = (0, 7)
\end{equation}

\begin{enumerate}[(a)]

\item ¿7 es una cota superior? Sí, porque todo elemento de A es menor que 7.

\item ¿0 es una cota inferior? Sí, porque todo elemento de A es mayor que 0.

\item ¿A está acotado superiormente? Sí, porque todo elemento es menor a 7.

\item ¿A está acotado inferiormente? Sí, porque todo elemento es mayor a 0.

\item Supremo, ínfimo, mínimo y máximo: en base al análisis inicial:

\begin{itemize}

\item $ \mathop{sup}(A) = 7 $

\item $ \mathop{max}(A) = \nexists $

\item $ \mathop{inf}(A) = 0 $

\item $ \mathop{min(A)} = \nexists $
 
\end{itemize}

Pese a que el conjunto está acotado, no hay mínimo ni máximo porque 0 y 7 no pertenecen al mismo.

\end{enumerate}

\subsectionx{7.D}

\begin{equation}
D = \mathbb{N}
\end{equation}

\begin{enumerate}[(a)]

\item ¿7 es una cota superior? No, el conjunto de los naturales no tiene cota superior. Para cualquier número natural $ n $, existe $ n + 1 $ mayor.

\item ¿0 es una cota inferior? Sí, porque todo número natural es mayor a cero. Nótese que se adopta la convención de que cero no pertene a $ \mathbb{N} $.

\item ¿A está acotado superiormente? No.

\item ¿A está acotado inferiormente? Sí, porque todo elemento es mayor a 0.

\item Supremo, ínfimo, mínimo y máximo: en base al análisis inicial:

\begin{itemize}

\item $ \mathop{sup}(A) = \nexists $

\item $ \mathop{max}(A) = \nexists $

\item $ \mathop{inf}(A) = 1 $

\item $ \mathop{min(A)} = 1 $
 
\end{itemize}

\end{enumerate}

\subsectionx{7.E}

\begin{equation}
E = \left\{ n - \frac{1}{n^2}, n \in \mathbb{N} \right\} = \left\{ \frac{n^3-1}{n^2}, n \in \mathbb{N} \right\}
\end{equation}

Analizando los valores:

\begin{equation}
E = \left\{ 0, \frac{7}{4}, \frac{26}{9}, \frac{63}{16}, \dots \right\}
\end{equation}

Analizando para valores de $ n $ grandes, en el numerador el -1 se torna despreciable, y simplificando, la expresión tiende a $ n $. Por ende, los valores tienden a $ +\infty $. 

\begin{enumerate}[(a)]

\item ¿7 es una cota superior? No, este conjunto no tiene cotas superiores.

\item ¿0 es una cota inferior? Sí, porque todo elemento es mayor o igual a cero.

\item ¿A está acotado superiormente? No.

\item ¿A está acotado inferiormente? Sí, porque todo elemento es mayor o igual a 0.

\item Supremo, ínfimo, mínimo y máximo: en base al análisis inicial:

\begin{itemize}

\item $ \mathop{sup}(A) = \nexists $

\item $ \mathop{max}(A) = \nexists $

\item $ \mathop{inf}(A) = 0 $

\item $ \mathop{min(A)} = 0 $
 
\end{itemize}

\end{enumerate}

\subsectionx{7.F}

\begin{equation}
F = \{ 1, 2, 3, 4 \}
\end{equation}

\begin{enumerate}[(a)]

\item ¿7 es una cota superior? Sí, porque todo elemento es menor a 7.

\item ¿0 es una cota inferior? Sí, porque todo elemento es mayor a cero.

\item ¿A está acotado superiormente? Sí.

\item ¿A está acotado inferiormente? Sí.

\item Supremo, ínfimo, mínimo y máximo: en base al análisis inicial:

\begin{itemize}

\item $ \mathop{sup}(A) = 4 $

\item $ \mathop{max}(A) = 4 $

\item $ \mathop{inf}(A) = 1 $

\item $ \mathop{min(A)} = 1 $
 
\end{itemize}

\end{enumerate}

\subsectionx{7.G}

\begin{equation}
G = \{ 5; 5,9; 5,99; 5,999; \dots \}
\end{equation}

Este conjunto está acotado entre 5 y 6. El valor 5 pertenece, pero 6 no, aunque los valores se aproximen infinitesimalmente.

\begin{enumerate}[(a)]

\item ¿7 es una cota superior? Sí, porque todo elemento es menor a 7.

\item ¿0 es una cota inferior? Sí, porque todo elemento es mayor a cero.

\item ¿A está acotado superiormente? Sí.

\item ¿A está acotado inferiormente? Sí.

\item Supremo, ínfimo, mínimo y máximo: en base al análisis inicial:

\begin{itemize}

\item $ \mathop{sup}(A) = 6 $

\item $ \mathop{max}(A) = \nexists $

\item $ \mathop{inf}(A) = 5 $

\item $ \mathop{min(A)} = 5 $
 
\end{itemize}

\end{enumerate}

\subsectionx{7.H}

\begin{equation}
H = \{ x \in \mathbb{R} / |x-2| < 1 \}
\end{equation}

Este conjunto comprende los números que están a una distancia menor a 1 respecto a 2. Por lo tanto, es el intervalo real $ (1, 3) $. Determinado esto, el análisis es trivial.

\begin{enumerate}[(a)]

\item ¿7 es una cota superior? Sí, porque todo elemento es menor a 7.

\item ¿0 es una cota inferior? Sí, porque todo elemento es mayor a cero.

\item ¿A está acotado superiormente? Sí.

\item ¿A está acotado inferiormente? Sí.

\item Supremo, ínfimo, mínimo y máximo: en base al análisis inicial:

\begin{itemize}

\item $ \mathop{sup}(A) = 3 $

\item $ \mathop{max}(A) = \nexists $

\item $ \mathop{inf}(A) = 1 $

\item $ \mathop{min(A)} = \nexists $
 
\end{itemize}

\end{enumerate}

\subsectionx{7.I}

\begin{equation}
I = \{ x \in \mathbb{R} / |x| > 3 \}
\end{equation}

Este conjunto comprende los números cuyo valor absoluto es mayor a 3. Por lo tanto, es es la unión de intervalos $ (-\infty, -3) \cup (3, +\infty) $. Determinado esto:

\begin{enumerate}[(a)]

\item ¿7 es una cota superior? No, porque el conjunto no está acotado superiormente.

\item ¿0 es una cota inferior? No, porque el conjunto no está acotado inferiormente.

\item ¿A está acotado superiormente? No.

\item ¿A está acotado inferiormente? No.

\item Supremo, ínfimo, mínimo y máximo: en base al análisis previo:

\begin{itemize}

\item $ \mathop{sup}(A) = \nexists $

\item $ \mathop{max}(A) = \nexists $

\item $ \mathop{inf}(A) = \nexists $

\item $ \mathop{min(A)} = \nexists $
 
\end{itemize}

\end{enumerate}

\sectionx{8}

\textbf{Considerar el conjunto B del ejercicio anterior.}

\begin{enumerate}[(a)]

\bfseries

\item Mostrar que 1 es cota superior de B.

\item Exhibir un elemento $ b $ de B que satisfaga $ 0,9 < b < 1 $.

\item Exhibir un elemento $ b $ de B que satisfaga $ 0,99 < b < 1 $.

\end{enumerate}

\hrule

\subsectionx{8.a}

\begin{equation}
B = \left\{ \frac{n}{n+1}, n \in \mathbb{N} \right\}
\end{equation}

Para hallar los valores de $ n $ para los que se satisface $ b_n = \frac{n}{n+1} < 1 $, se plantea:

\begin{equation}
\frac{n}{n + 1} < 1
\end{equation}

Como $ n $ es siempre positivo por ser natural, es válido multiplicar miembro a miembro por $(n+1)$.

\begin{equation}
n < n + 1
\end{equation}

Esto se satisface para todo $ n $ natural, por lo tanto 1 es una cota superior del conjunto $ B $.

\subsectionx{8.b}

Si se exige $ b_n > 0,9 $, se hallará el menor valor de $ n $ que satisfaga lo pedido.

\begin{equation}
\frac{n}{n+1} > 0,9 \Rightarrow n > 0,9 n + 0,9 \Rightarrow 0,1 n > 0,9 \Rightarrow n > 9
\end{equation}

Para $ n = 10 $, resulta $ \hresult{8.b}{ b = \frac{10}{11} = 0,909090\dots } $

\subsectionx{8.c}

Planteando $ b_n > 0,99 $ y operando igual que en el inciso previo, resulta $ n > 99 $.

Para $ n = 100 $, resulta $ \Rightarrow \hresult{8.c}{ b = \frac{100}{101} = 0,990990\dots } $

\begin{equation}
\frac{n}{n+1} > 0,9 \Rightarrow n > 0,9 n + 0,9 \Rightarrow 0,1 n > 0,9 \Rightarrow n > 9
\end{equation}

Para $ n = 10 $, resulta $ \hresult{8.b}{ b = \frac{10}{11} = 0,909090\dots } $

\sectionx{9}

\textbf{Considerar el conjunto $ P = \left\{ \frac{2n-1}{n+2}, n \in \mathbb{N} \right\} $}

\begin{enumerate}[(a)]

\bfseries

\item Mostrar que 2 es una cota superior de P.

\item Exhibir un elemento $p$ de P que satisfaga $ 1,99 < p < 2 $.

\item Mostrar que si $ t < 2 $, existe un elemento $ p $ de P que satisface $ t < p < 2 $. Deducir entonces que $ \mathop{\text{sup}}(P) = 2 $.

\end{enumerate}

\hrule

\subsectionx{9.a}

¿Para qué valores de $ n $ se satisface $ p_n \leq 2 $?

\begin{equation}
\frac{2n-1}{n+2} \leq 2
\end{equation}

Siendo $ n $ un número natural, $ (n+2) > 0 $ para todo $ n $ y por ende multiplicar miembro a miembro no altera el sentido de la desigualdad.

\begin{equation}
2n-1 \leq 2n + 4 \Rightarrow -1 \leq 4
\end{equation}

Esto se cumple para todo valor de $ n $, y por ende todo elemento de $ P $ es menor o igual a 2, lo cual satisface la definición de cota superior.

\subsectionx{9.b}

En el inciso anterior, cambiar $ \leq $ por $ < $ no cambia nada; todo elemento de $ P $ es estrictamente menor a 2, por lo cual esa restricción ya está cubierta. Considérese qué valores de $ n $ satisfacen $ p_n > 1,99 $.

\begin{equation}
1,99 < \frac{2n-1}{n+2}
\end{equation}

Multiplicando miembro a miembro por $ (n+2) $:

\begin{equation}
1,99 n + 3,98 < 2n - 1 \Rightarrow 4,98 < 0,01 n \Rightarrow 498 < n
\end{equation}

Hay infinitos elementos de P que satisfacen lo pedido, pero el mínimo es el correspondiente a $ n = 499 $.

\begin{equation}
\hresult{9.b}{ p = p_{499} \approx 1,9900 }
\end{equation}

\subsectionx{9.c}

De manera general, considérese para qué valores de $ n $ se cumple $ p_n > t $.

\begin{subequations}
\begin{align}
& t < \frac{2n-1}{n+2} \\
& t (n+2) < 2n-1 \\
& t \cdot n + 2t < 2n -1 \\
\end{align}
\end{subequations}

Llegado este punto, se utiliza el hecho de que $ t < 2 $ para elegir de qué lado acomodar los términos con $ n $. Para que el coeficiente quede positivo, se resta $ t \cdot n $ miembro a miembro.

\begin{equation}
2t + 1 < (2-t) n \Rightarrow \frac{2t+1}{2-t} < n
\end{equation}

Nótese que en el último paso, se utilizó el hecho de que $ t < 2 \Rightarrow 0 < 2-t $ para poder dividir miembro a miembro sin alterar la desigualdad. Por ende, para cualquier valor de $ t < 2$, existirá $ p = \lceil \frac{2t+1}{2-t} \rceil $ tal que $ t < p < 2$. En otras palabras, los elementos de $ P $ se acercan infinitesimalmente a 2 al crecer n. Por ende, 2 es la mínima cota superior de P, lo que la convierte en el supremo del conjunto. Nótese que por los mismos motivos, P no tiene un máximo.

\sectionx{10}

\textbf{Mostrar que existe un número natural $ n $ que satisface $ \frac{1}{n} < 0,001 $. En general, mostrar que cualquiera sea $ x $ positivo, existe un número natural $ n $ que satisface $ \frac{1}{n} < x $. Deducir de ello que $ \mathop{\text{inf}}\left\{ \frac{1}{n}, n \in \mathbb{N} \right\} = 0$}

\vspace{1em}
\hrule
\vspace{1em}

En primer lugar, se calcularán los valores de $ n $ que satisfacen $ \frac{1}{n} < 0,001 $.

\begin{equation}
\frac{1}{n} < 0,001
\end{equation}

Siendo $ n $ natural, es positivo y multiplicar miembro a miembro por $ n $ no altera la desigualdad.

\begin{equation}
1 < 0,001 n \Rightarrow \frac{1}{0,001} < n \Rightarrow 1000 < n
\end{equation}

Por lo tanto, cualquier valor de $ n $ mayor a 1000 satisface la desigualdad. De manera general, considérese un valor $ x $ positivo.

\begin{equation}
\frac{1}{n} < x \Rightarrow 1 < n \cdot x
\end{equation}

En este punto, se utiliza el hecho de que $ x > 0 $ para multiplicar miembro a miembro por $ \frac{1}{x} $ sin alterar la desigualdad.

\begin{equation}
\frac{1}{x} < n
\end{equation}

Esto establece el hecho de que la expresión $ \frac{1}{n} $ puede tomar valores arbitrariamente cercanos a cero. Agregando a esto el hecho de que $ \frac{1}{n} > 0 $ para todo $ n $ por ser $ n $ positivo, resulta entonces que 0 es la mínima cota inferior del conjunto y por ende es el ínfimo.

\sectionx{10}

\textbf{Sean A y B dos conjuntos de números reales no vacíos y acotados de modo que $ A \in B $. Ordenar de menor a mayor los siguientes números:}

\begin{equation}
\mathop{\text{sup}} A, \mathop{\text{sup}} B, \mathop{\text{inf}} A, \mathop{\text{inf}} B
\end{equation}

\textbf{Exhibir un ejemplo donde $ \mathop{\text{sup}} A = \mathop{\text{sup}} B $ y otro donde la desigualdad sea estricta}

\vspace{1em}
\hrule
\vspace{1em}

Si $ A \in B $, tiene que cumplirse lo siguiente:

\begin{itemize}
\item $ \mathop{\text{inf}} B \leq \mathop{\text{inf}} A $: Porque A no puede tener elementos menores al ínfimo de B. 
\item $ \mathop{\text{sup}} A \leq \mathop{\text{sup}} B $: Porque A no puede tener elementos mayores al supremo de B.
\item $ \mathop{\text{inf}} A \leq \mathop{\text{sup A}} $: Para cualquier conjunto, el ínfimo no puede superar al supremo.
\end{itemize}

Esto garantiza que todo elemento de A esté contenido en B. Ahora bien, las desigualdades pueden darse en diferentes combinaciones. El orden general siempre será el mismo, pero algunos valores pueden coincidir.

\begin{itemize}

\item Caso 1: Sin coincidencias.

\begin{equation}
\mathop{\text{inf}} B < \mathop{\text{inf}} A < \mathop{\text{sup}} A < \mathop{\text{sup}} B.
\end{equation}

Ejemplo: $ A = (-1,1), B = (-3, 3) $

\item Caso 2: Coinciden los ínfimos de A y B.

\begin{equation}
\mathop{\text{inf}} B = \mathop{\text{inf}} A < \mathop{\text{sup}} A < \mathop{\text{sup}} B.
\end{equation}

Ejemplo: $ A = (-3,0), B = (-3,3) $

\item Caso 3: Coinciden los supremos de A y B.

\begin{equation}
\mathop{\text{inf}} B < \mathop{\text{inf}} A < \mathop{\text{sup}} A = \mathop{\text{sup}} B.
\end{equation}

Ejemplo: $ A = (0,3), B = (-3,3) $

\item Caso 4: A y B comparten tanto ínfimo como supremo, pero no son el mismo conjunto.

\begin{equation}
\mathop{\text{inf}} B = \mathop{\text{inf}} A < \mathop{\text{sup}} A = \mathop{\text{sup}} B.
\end{equation}

Ejemplo: $ A = (-3,3), B = [-3,3] $

\item Caso 5: A tiene el mismo ínfimo y supremo (es un punto), y ninguno coincide con los de B.

\begin{equation}
\mathop{\text{inf}} B < \mathop{\text{inf}} A = \mathop{\text{sup}} A < \mathop{\text{sup}} B.
\end{equation}

Ejemplo: $ A = \{0\}, B = (-3,3) $

\item Caso 6: A tiene el mismo ínfimo y supremo (es un punto), y su ínfimo coincide con el de B.

\begin{equation}
\mathop{\text{inf}} B = \mathop{\text{inf}} A = \mathop{\text{sup}} A < \mathop{\text{sup}} B.
\end{equation}

Ejemplo: $ A = \{-3\}, B = (-3,3) $

\item Caso 7: A tiene el mismo ínfimo y supremo (es un punto), y su supremo coincide con el de B.

\begin{equation}
\mathop{\text{inf}} B < \mathop{\text{inf}} A = \mathop{\text{sup}} A = \mathop{\text{sup}} B.
\end{equation}

Ejemplo: $ A = \{3\}, B = (-3,3) $

\end{itemize}

\sectionx{12}

\textbf{Determinar, en caso de que existan, el supremo, el ínfimo, el máximo y el mínimo de los siguientes conjuntos:}

\begin{enumerate}[(a)]

\item $ A = \{ x \in \mathbb{R} / x^2 - 3x + 2 < 0 \} $

\item $ B = \{ y = x^2 - 3x + 2, x \in (0, 2) \} $

\item $ C = \{ y = x^2 - 3x + 2, x \in \mathbb{R} \} $

\end{enumerate}

\hrule
\vspace{1em}

Considerando que los 3 conjuntos están definidos en base a la misma función cuadrática, conviene analizarla. Aplicando la fórmula para las raíces, vale decir:

\begin{equation}
x = \frac{-b \pm \sqrt{b^2-4ac}}{2a}
\end{equation}

Se obtienen las raíces reales y diferentes $ x_1 = 2, x_2 = 1 $. Por ende:

\begin{equation}
x^2 - 3x + 2 = (x-2) (x-1)
\end{equation}

Esto servirá para analizar el signo. Para analizar la imagen, conviene pasar a la forma canónica. Completando el cuadrado, resulta:

\begin{equation}
x^2 - 3x + 2 = \left( x- \frac{3}{2} \right)^2 - \frac{1}{4}
\end{equation}

\subsectionx{12.a}

\begin{equation}
A = \{ x \in \mathbb{R} / x^2 - 3x + 2 < 0 \}
\end{equation}

Para esto conviene usar la forma factorizada.

\begin{equation}
(x-2) (x-1) < 0
\end{equation}

Caso 1: primer factor positivo y segundo factor negativo.

\begin{subequations}
\begin{align}
& x-2 < 0 \Rightarrow x < 2 \\
& x-1 > 0 \Rightarrow x > 1
\end{align}
\end{subequations}

Ergo, este caso corresponde al conjunto $ (1, 2) $.

Caso 2: primer factor negativo y segundo factor positivo.

\begin{subequations}
\begin{align}
& x-2 > 0 \Rightarrow x > 2 \\
& x-1 < 0 \Rightarrow x < 1
\end{align}
\end{subequations}

Este caso correponde al conjunto vacío, dado que no pueden darse a la vez $ x > 2 \wedge x < 1 $. Por lo tanto, resulta:

\begin{equation}
A = \{ x \in \mathbb{R} / x^2 - 3x + 2 < 0 \} = (1, 2)
\end{equation}

Por lo tanto:

\begin{itemize}

\item $ \hresult{}{ \mathop{\text{inf}} A = 1 } $
\item $ \hresult{}{ \mathop{\text{min}} A = \nexists } $
\item $ \hresult{}{ \mathop{\text{max}} A = \nexists } $
\item $ \hresult{}{ \mathop{\text{sup}} A = 2 } $

\end{itemize}

\subsectionx{12.b}

\begin{equation}
B = \{ y = x^2 - 3x + 2, x \in (0, 2) \}
\end{equation}

El conjunto B sería entonces la imagen del intervalo (0, 2) a través de la función cuadrática. La forma más directa de visualizar esto es graficarla. Para ello conviene usar la forma canónica calculada previamente:

\begin{equation}
x^2 - 3x + 2 = \left( x- \frac{3}{2} \right)^2 - \frac{1}{4}
\end{equation}

Por ende, el vértice de la parábola se halla en el punto $ \left( -\frac{3}{2}, -\frac{1}{4} \right) $. Sabiendo esto y que las raíces son $ x_1 = 2, x_2 = 1 $, el gráfico resulta:

\figurex{Ejercicio 12.b}{2.0}{../img/guide_02/ex_12b.png}{fig:12b} 

Resulta entonces, observando los valores que toma la función en el intervalo (0, 2), $ B = \left[ -\frac{1}{4}, 2 \right) $. Nótese que el mínimo se incluye porque se alcanza dentro del intervalo, y el máximo no entra porque se alcanza en el extremo del intervalo, que no pertenece al conjunto. Por eso se omite el 2. Ergo:

\begin{itemize}

\item $ \hresult{}{ \mathop{\text{inf}} B = -\frac{1}{4} } $
\item $ \hresult{}{ \mathop{\text{min}} B = -\frac{1}{4} } $
\item $ \hresult{}{ \mathop{\text{max}} B = \nexists } $
\item $ \hresult{}{ \mathop{\text{sup}} B = 2 } $

\end{itemize}

\subsectionx{12.c}

\begin{equation}
C = \{ y = x^2 - 3x + 2, x \in \mathbb{R} \}
\end{equation}

Ahora se pide el conjunto imagen completo. Observando el gráfico, resulta:

\begin{equation}
C = \left[ -\frac{1}{4}, +\infty \right)
\end{equation}

Por ende:

\begin{itemize}

\item $ \hresult{}{ \mathop{\text{inf}} C = -\frac{1}{4} } $
\item $ \hresult{}{ \mathop{\text{min}} C = -\frac{1}{4} } $
\item $ \hresult{}{ \mathop{\text{max}} C = \nexists } $
\item $ \hresult{}{ \mathop{\text{sup}} C = \nexists } $

\end{itemize}

\end{document}
