\documentclass{article}

%\usepackage{afterpage}
\usepackage{amsmath}
\usepackage{amssymb}
%\usepackage{cancel}
%\usepackage{graphicx}
\usepackage[spanish]{babel}
\usepackage{enumerate}
\usepackage{hyperref}
\hypersetup{
    colorlinks,
    citecolor=black,
    filecolor=black,
    linkcolor=black,
    urlcolor=black,
}
%\usepackage{longdivision}
%\usepackage{polynom}
\usepackage{tcolorbox}
%\usepackage{xcolor}
%
\tcbuselibrary{theorems}
%
\newcommand{\hresult}[2]{\tcboxmath[colback=orange!25!white,colframe=orange, title=#1] {#2} }
%\newcommand{\hresulte}[3]{\tcboxmath[colback=orange!25!white,colframe=orange, title=#1] { \hspace{#3} #2 \hspace{#3} } }
\newcommand{\figurex}[4]{\begin{figure}[ht] \caption{#1} \includegraphics[scale=#2]{#3} \centering \label{#4}\end{figure}}
%\newcommand{\figurexnp}[4]{\afterpage{\figurex{#1}{#2}{#3}{#4}}}
\newcommand{\sectionx}[1]{\section*{#1}\label{sec:#1}\addcontentsline{toc}{section}{\nameref{sec:#1}}}
\newcommand{\subsectionx}[1]{\subsection*{#1}\label{subsec:#1}\addcontentsline{toc}{subsection}{\nameref{subsec:#1}}}

\renewcommand{\Bbb}{\mathbb}

\title{Ejercicios de Análisis Matemático CBC (28) \\
Práctica 2: La Recta real \\
Cátedra Ruiz, curso 62802 \\
1° C 2003}
\author{Darío Eduardo Ramos}

\begin{document}
\maketitle

\tableofcontents{}

\newpage

\sectionx{1}

\textbf{Representar en la recta numérica:}

\begin{enumerate}[(a)]

\bfseries

\item $ -5; -1; 3; 6; \frac{3}{8}; 1+\frac{2}{5}; 1-\frac{2}{5}; -\sqrt{2}; \sqrt{2}+1; \sqrt{2}-1; -\sqrt{2}+1; -\sqrt{2}-1 $

\item $ -3; -2; -1; 0; 1; 2; 3; -\pi; -\frac{\pi}{2}; \frac{\pi}{2}; \pi; \frac{3}{2}\pi; 3,14; -3,14 $

\end{enumerate}

\hrule
\vspace{1em}

Graficar números reales en la recta equivale a ordenarlos. Con saber los valores de cada uno, aproximando si hace falta, es suficiente. Para las fracciones, siempre se puede obtener un valor exacto, salvo que sea un decimal periódico. Para las expresiones irracionales, habrá que tomar una cantidad suficiente de decimales que permita la comparación. Con todo esto en mente, el resultado será presentado como el orden final de los números, ya que el gráfico es trivial.

\subsectionx{1.a}

Valores de menor a mayor:

\begin{enumerate}[(I)]

\item $ -5 $

\item $ -\sqrt{2}-1 \approx -2,41 \dots $

\item $ -\sqrt{2} \approx -1,41 \dots $

\item $ -1 $

\item $ -\sqrt{2}+1 \approx -0,41 \dots $

\item $ \frac{3}{8} = 0,375 $

\item $ \sqrt{2}-1 \approx 0,41 \dots $

\item $ 1-\frac{2}{5} = \frac{3}{5} = 0,6 $

\item $ 1+\frac{2}{5} = \frac{7}{5} = 1,4 $

\item $ \sqrt{2}+1 \approx 2,41 \dots $

\item $ 3 $

\item $ 6 $

\end{enumerate}

\subsectionx{1.b}

Valores de menor a mayor:

\begin{enumerate}[(I)]

\item $ -\pi \approx -3,1415 \dots $

\item $ -3,14 $

\item $ -3 $

\item $ -2 $

\item $ -\frac{\pi}{2} \approx -1,5707 \dots $

\item $ -1 $

\item $ 0 $

\item $ 1 $

\item $ \frac{\pi}{2} \approx 1,5707 \dots $

\item $ 2 $

\item $ 3 $

\item $ 3,14 $

\item $ \pi \approx 3,1415 \dots $

\end{enumerate}


\sectionx{2}

\textbf{Represente en la recta numérica los siguientes conjuntos. Escribirlos como intervalos o unión de intervalos.}


\begin{enumerate}[(a)]

\bfseries

\item Todos los números reales mayores que -1.

\item Todos los números reales menores o iguales que 2.

\item Todos los números reales que distan del 0 menos que 3.

\item $ \{ x \in \mathbb{R} / 2x - 3 > 5 \} $

\item $ \{ x \in \mathbb{R} / -3 < x \leq 3 \} $

\item $ \{ x \in \mathbb{R} / 1 < 2x-3 < 5 \} $

\item $ \{ x \in \mathbb{R} / x(2x-3) > 0 \} $

\item $ \{ x \in \mathbb{R} / x^2 - 36 < 0 \} $

\item $ \{ x \in \mathbb{R} / x^3 - x < 0 \} $

\item $ \{ x \in \mathbb{R} / 1 + \frac{2}{x} < 3 \} $

\item $ \{ x \in \mathbb{R} / \frac{1}{x} < \frac{4}{x} \} $

\item $ \{ x \in \mathbb{R} / |x| < 3 \} $

\item $ \{ x \in \mathbb{R} / |x-2| < 3 \} $

\item $ \{ x \in \mathbb{R} / |x+2| < 3 \} $

\item $ \{ x \in \mathbb{R} / |x| > 3 \} $

\end{enumerate}
\hrule
\vspace{1em}

Representar gráficamente un conjunto en la recta real no tiene mucho más misterio que colorear un intervalo o unión de intervalos. Para representar que un extremo no se incluye por desigualdad estricta, se suele dibujar un círculo vacío alrededor del punto a excluir. Cuando la desigualdad no es estricta, se pinta el interior del círculo. Por brevedad, se harán algunos gráficos como ejemplos y se omitirá el resto.

\subsectionx{2.a}

Si se denomina $A$ al conjunto pedido, expresarlo como intervalo es directo:

\begin{equation}
\hresult{2.a}{ A = (-1, +\infty) }
\end{equation}

\figurex{Ejercicio 2.a}{5}{../img/guide_02/ex_02a.png}{fig:2a}

\subsectionx{2.b}

\begin{equation}
\hresult{2.b}{ A = (-\infty, 2] }
\end{equation}

\figurex{Ejercicio 2.b}{5}{../img/guide_02/ex_02b.png}{fig:2b}

\subsectionx{2.c}

\begin{equation}
\hresult{2.c}{ A = (-3, 3) }
\end{equation}

\subsectionx{2.d}

En este caso hay que transformar la desigualdad un poco:

\begin{subequations}
\begin{align}
& 2x-3 > 5 \\
& 2x > 8 \\
& x > 4
\end{align}
\end{subequations}

\begin{equation}
\hresult{2.d}{ A = (4, +\infty) }
\end{equation}

\subsectionx{2.e}

\begin{equation}
\hresult{2.e}{ A = (-3, 3] }
\end{equation}

\subsectionx{2.f}

\begin{subequations}
\begin{align}
& 1 < 2x-3 < 5 \\
& 4 < 2x < 8 \\
& 2 < x < 4
\end{align}
\end{subequations}

\begin{equation}
\hresult{2.f}{ A = (2, 4) }
\end{equation}

\subsectionx{2.g}

En este caso se tiene un producto, por lo que hay que considerar dos casos usando la regla de los signos.

En primer lugar, que ambos factores sean positivos y no nulos:

\begin{equation}
x > 0 \wedge 2x-3 > 0 \Rightarrow x > 0 \wedge x > \frac{3}{2} \Rightarrow x \in \left( \frac{3}{2}, +\infty \right)
\end{equation}

En segundo y último lugar, que ambos factores sean negativos a la vez.

\begin{equation}
x < 0 \wedge 2x-3 < 0 \Rightarrow x < 0 \wedge x < \frac{3}{2} \Rightarrow x \in (-\infty, 0)
\end{equation}

El resultado final es la unión de los intervalos asociados a cada caso:

\begin{equation}
\hresult{2.g}{ A = (-\infty, 0) \cup \left( \frac{3}{2}, +\infty \right) }
\end{equation}

\subsectionx{2.h}

Recuérdese que cuando se eleva un valor al cuadrado y luego se toma la raíz cuadrada, no se puede cancelar raíz con cuadrado sin considerar el módulo. Esto es porque al elevar los valores negativos al cuadrado, el resultado es positivo y puede tomarse la raíz.

\begin{subequations}
\begin{align}
& x^2 - 36 < 0 \\
& x^2 < 36 \\
& \sqrt{x^2} < 6 \\
& |x| < 6 \\
& -6 < x < 6
\end{align}
\end{subequations}

\begin{equation}
\hresult{2.h}{ A = (-6, 6) }
\end{equation}

\subsectionx{2.i}

Siempre que se analiza el signo, conviene llevar la expresión a un producto de dos factores.

\begin{subequations}
\begin{align}
& x^3 - x < 0 \\
& x (x^2-1) < 0
\end{align}
\end{subequations}

Caso 1: primer factor positivo, segundo negativo.

\begin{equation}
x > 0 \wedge x^2-1 < 0 \Rightarrow x > 0 \wedge |x| < 1  \Rightarrow x \in (0, 1)
\end{equation}

Caso 2: primer factor negativo, segundo positivo.

\begin{equation}
x < 0 \wedge x^2-1 > 0 \Rightarrow x < 0 \wedge |x| > 1  \Rightarrow x \in (-\infty, -1)
\end{equation}

\begin{equation}
\hresult{2.i}{ A = (-\infty, -1) \cup (0,1) }
\end{equation}

\subsectionx{2.j}

\begin{subequations}
\begin{align}
& 1 + \frac{2}{x} < 3 \\
& \frac{2}{x} < 2 \\
& \frac{1}{x} < 1
\end{align}
\end{subequations}

Llegado este punto, no se puede multiplicar miembro a miembro por $ x $ directamente, porque si $ x $ es negativo hay que invertir la desigualdad. Por lo tanto, se consideran ambos casos:

Caso 1: x positivo.

\begin{equation}
x > 0 \wedge 1 < x \Rightarrow x \in (1, +\infty)
\end{equation}

Caso 2: x negativo.

\begin{equation}
x < 0 \wedge 1 > x \Rightarrow x \in (-\infty, 0)
\end{equation}

\begin{equation}
\hresult{2.j}{ A = (-\infty, 0) \cup (1, +\infty) }
\end{equation}

\subsectionx{2.k}

Nuevamente, hay que considerar el signo de $ x $.

Caso 1: x positivo.

\begin{equation}
x > 0 \wedge \frac{1}{x} < \frac{4}{x} \Rightarrow x > 0 \wedge 1 < 4 \Rightarrow x \in (0, +\infty)
\end{equation}

Caso 2: x negativo.

\begin{equation}
x < 0 \wedge \frac{1}{x} < \frac{4}{x} \Rightarrow x < 0 \wedge 1 > 4 \Rightarrow x \in \emptyset
\end{equation}

Como el caso 2 no aporta elementos, resulta:

\begin{equation}
\hresult{2.k}{ A = (0, +\infty) }
\end{equation}

\subsectionx{2.l}

De manera general, $ |x| < k \Rightarrow -k < x < k $. Por lo tanto:

\begin{equation}
\hresult{2.l}{ A = (-3, 3) }
\end{equation}

\subsectionx{2.m}

Hay que analizar el signo del argumento del módulo.

Caso 1: $ x-2 \geq 0 \Rightarrow x - 2 < 3 \Rightarrow x \geq 2 \wedge x < 5 \Rightarrow x \in [2, 5) $.

Caso 2: $ x - 2 < 0 \Rightarrow -x+2 < 3 \Rightarrow x < 2 \wedge x > -1 \Rightarrow x \in (-1, 2)  $

Combinando los dos intervalos, resulta entonces:

\begin{equation}
\hresult{2.m}{ A = (-1, 5) }
\end{equation}

\subsectionx{2.n}

Caso 1: $ x+2 \geq 0 \Rightarrow x + 2 < 3 \Rightarrow x \geq -2 \wedge x < 1 \Rightarrow x \in [-2, 1) $.

Caso 2: $ x+2 < 0 \Rightarrow -x-2 < 3 \Rightarrow x < -2 \wedge x > -5 \Rightarrow x \in (-5, -2)  $

\begin{equation}
\hresult{2.n}{ A = (-5, 1) }
\end{equation}

\subsectionx{2.ñ}

De manera general, para $ k \in \mathbb{R} \wedge k > 0 $, $ |x| > k \Rightarrow x > k \wedge x < -k \Rightarrow x \in \{ (-\infty, -k) \cup (k, +\infty) \} $,.

\begin{equation}
\hresult{2.ñ}{ A = (-\infty, -3) \cup (3, +\infty) }
\end{equation}

\end{document}
