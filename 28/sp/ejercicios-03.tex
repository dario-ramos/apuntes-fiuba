\documentclass{article}

\usepackage{amsmath}
\usepackage{amssymb}
\usepackage[spanish]{babel}
\usepackage{enumerate}
\usepackage{hyperref}
\hypersetup{
    colorlinks,
    citecolor=black,
    filecolor=black,
    linkcolor=black,
    urlcolor=black,
}
\usepackage{tcolorbox}

\tcbuselibrary{theorems}

\newcommand{\hresult}[2]{\tcboxmath[colback=orange!25!white,colframe=orange, title=#1] {#2} }
\newcommand{\figurex}[4]{\begin{figure}[ht] \caption{#1} \includegraphics[scale=#2]{#3} \centering \label{#4}\end{figure}}
\newcommand{\sectionx}[1]{\section*{#1}\label{sec:#1}\addcontentsline{toc}{section}{\nameref{sec:#1}}}
\newcommand{\subsectionx}[1]{\subsection*{#1}\label{subsec:#1}\addcontentsline{toc}{subsection}{\nameref{subsec:#1}}}

\renewcommand{\Bbb}{\mathbb}

\title{Ejercicios de Análisis Matemático CBC (28) \\
Práctica 3: Sucesiones \\
Cátedra Ruiz, curso 62802 \\
1° C 2003}
\author{Darío Eduardo Ramos}

\begin{document}
\maketitle

\tableofcontents{}

\newpage

\sectionx{1}

\textbf{Escribir los primeros cinco términos de las siguientes sucesiones:}

\begin{enumerate}[(a)]

\bfseries

\item $ a_n = \frac{\sqrt{n}}{n+1} $

\item $ b_n = \frac{2^{n-1}}{(2n-1)^3} $

\item $ c_n = \frac{(-1)^{n+1}}{n!} $

\item $ d_n = \frac{\cos(n\pi)}{n} $

\end{enumerate}

\hrule
\vspace{1em}

Evaluar el término $ a_i $ equivale a hacer $ n = i $ en la expresión del término general $ a_n $ y calcular el valor. En este caso, se piden $ a_1, a_2, \dots, a_5 $. Se asume la convención de que todas las sucesiones comienzan en $ n = 1 $.

\subsectionx{1.a}

\begin{equation}
a_n = \frac{\sqrt{n}}{n+1}
\end{equation}

\begin{subequations}
\begin{align}
& \hresult{}{a_1 = \frac{1}{2} = 0,5000 } \\
& \hresult{}{a_2 = \frac{\sqrt{2}}{3} \approx 0,47140 } \\
& \hresult{}{a_3 = \frac{\sqrt{3}}{4} \approx 0,43301 } \\
& \hresult{}{a_4 = \frac{\sqrt{4}}{5} =  0,4000 } \\
& \hresult{}{a_5 = \frac{\sqrt{5}}{6} =  0,37268 }
\end{align}
\end{subequations}

\subsectionx{1.b}

\begin{equation}
b_n = \frac{2^{n-1}}{(2n-1)^3}
\end{equation}

\begin{subequations}
\begin{align}
& \hresult{}{b_1 = \frac{2^0}{1^3} = \frac{1}{1} = 1,0000 } \\
& \hresult{}{b_2 = \frac{2^1}{3^3} = \frac{2}{27} \approx 0,074074 } \\
& \hresult{}{b_3 = \frac{2^2}{5^3} = \frac{4}{125} = 0,032000 } \\
& \hresult{}{b_4 = \frac{2^3}{7^3} = \frac{8}{343} \approx 0,023323 } \\
& \hresult{}{b_5 = \frac{2^4}{9^3} = \frac{16}{729} \approx 0,021948 }
\end{align}
\end{subequations}

\subsectionx{1.c}

\begin{equation}
c_n = \frac{(-1)^{n+1}}{n!}
\end{equation}

\begin{subequations}
\begin{align}
& \hresult{}{c_1 = \frac{1}{1!} = \frac{1}{1} = 1,0000 } \\
& \hresult{}{c_2 = \frac{-1}{2!} = -\frac{1}{2} = 0,5000 } \\
& \hresult{}{c_3 = \frac{1}{3!} = \frac{1}{6} \approx 0,16667 } \\
& \hresult{}{c_4 = \frac{-1}{4!} = -\frac{1}{24} \approx -0,041667 } \\
& \hresult{}{c_5 = \frac{1}{5!} = \frac{1}{120} \approx 0,008333 }
\end{align}
\end{subequations}

\subsectionx{1.d}

\begin{equation}
d_n = \frac{\cos(n\pi)}{n}
\end{equation}

\begin{subequations}
\begin{align}
& \hresult{}{d_1 = \frac{ \cos(\pi) }{1} = -\frac{1}{1} = -1,0000 } \\
& \hresult{}{d_2 = \frac{ \cos(2\pi) }{2} = \frac{1}{2} = 0,5000 } \\
& \hresult{}{d_3 = \frac{ \cos(3\pi) }{3} = -\frac{1}{3} \approx 0,33333 } \\
& \hresult{}{d_4 = \frac{ \cos(4\pi) }{4} = \frac{1}{4} = 0,25000 } \\
& \hresult{}{d_5 = \frac{ \cos(5\pi) }{5} = -\frac{1}{5} = 0,20000 0,008333 }
\end{align}
\end{subequations}

\end{document}
