\documentclass{article}

\usepackage{amsmath}
\usepackage{amssymb}
\usepackage[spanish]{babel}
\usepackage{enumerate}
\usepackage{hyperref}
\hypersetup{
    colorlinks,
    citecolor=black,
    filecolor=black,
    linkcolor=black,
    urlcolor=black,
}
\usepackage{tcolorbox}

\tcbuselibrary{theorems}

\newcommand{\hresult}[2]{\tcboxmath[colback=orange!25!white,colframe=orange, title=#1] {#2} }
\newcommand{\figurex}[4]{\begin{figure}[ht] \caption{#1} \includegraphics[scale=#2]{#3} \centering \label{#4}\end{figure}}
\newcommand{\sectionx}[1]{\section*{#1}\label{sec:#1}\addcontentsline{toc}{section}{\nameref{sec:#1}}}
\newcommand{\subsectionx}[1]{\subsection*{#1}\label{subsec:#1}\addcontentsline{toc}{subsection}{\nameref{subsec:#1}}}
\newcommand{\limninf}{\lim_{n \rightarrow +\infty}}

\renewcommand{\Bbb}{\mathbb}

\title{Ejercicios de Análisis Matemático CBC (28) \\
Práctica 3: Sucesiones \\
Cátedra Ruiz, curso 62802 \\
1° C 2003}
\author{Darío Eduardo Ramos}

\begin{document}
\maketitle

\tableofcontents{}

\newpage

\sectionx{1}

\textbf{Escribir los primeros cinco términos de las siguientes sucesiones:}

\begin{enumerate}[(a)]

\bfseries

\item $ a_n = \frac{\sqrt{n}}{n+1} $

\item $ b_n = \frac{2^{n-1}}{(2n-1)^3} $

\item $ c_n = \frac{(-1)^{n+1}}{n!} $

\item $ d_n = \frac{\cos(n\pi)}{n} $

\end{enumerate}

\hrule
\vspace{1em}

Evaluar el término $ a_i $ equivale a hacer $ n = i $ en la expresión del término general $ a_n $ y calcular el valor. En este caso, se piden $ a_1, a_2, \dots, a_5 $. Se asume la convención de que todas las sucesiones comienzan en $ n = 1 $.

\subsectionx{1.a}

\begin{equation}
a_n = \frac{\sqrt{n}}{n+1}
\end{equation}

\begin{subequations}
\begin{align}
& \hresult{}{a_1 = \frac{1}{2} = 0,5000 } \\
& \hresult{}{a_2 = \frac{\sqrt{2}}{3} \approx 0,47140 } \\
& \hresult{}{a_3 = \frac{\sqrt{3}}{4} \approx 0,43301 } \\
& \hresult{}{a_4 = \frac{\sqrt{4}}{5} =  0,4000 } \\
& \hresult{}{a_5 = \frac{\sqrt{5}}{6} =  0,37268 }
\end{align}
\end{subequations}

\subsectionx{1.b}

\begin{equation}
b_n = \frac{2^{n-1}}{(2n-1)^3}
\end{equation}

\begin{subequations}
\begin{align}
& \hresult{}{b_1 = \frac{2^0}{1^3} = \frac{1}{1} = 1,0000 } \\
& \hresult{}{b_2 = \frac{2^1}{3^3} = \frac{2}{27} \approx 0,074074 } \\
& \hresult{}{b_3 = \frac{2^2}{5^3} = \frac{4}{125} = 0,032000 } \\
& \hresult{}{b_4 = \frac{2^3}{7^3} = \frac{8}{343} \approx 0,023323 } \\
& \hresult{}{b_5 = \frac{2^4}{9^3} = \frac{16}{729} \approx 0,021948 }
\end{align}
\end{subequations}

\subsectionx{1.c}

\begin{equation}
c_n = \frac{(-1)^{n+1}}{n!}
\end{equation}

\begin{subequations}
\begin{align}
& \hresult{}{c_1 = \frac{1}{1!} = \frac{1}{1} = 1,0000 } \\
& \hresult{}{c_2 = \frac{-1}{2!} = -\frac{1}{2} = 0,5000 } \\
& \hresult{}{c_3 = \frac{1}{3!} = \frac{1}{6} \approx 0,16667 } \\
& \hresult{}{c_4 = \frac{-1}{4!} = -\frac{1}{24} \approx -0,041667 } \\
& \hresult{}{c_5 = \frac{1}{5!} = \frac{1}{120} \approx 0,008333 }
\end{align}
\end{subequations}

\subsectionx{1.d}

\begin{equation}
d_n = \frac{\cos(n\pi)}{n}
\end{equation}

\begin{subequations}
\begin{align}
& \hresult{}{d_1 = \frac{ \cos(\pi) }{1} = -\frac{1}{1} = -1,0000 } \\
& \hresult{}{d_2 = \frac{ \cos(2\pi) }{2} = \frac{1}{2} = 0,5000 } \\
& \hresult{}{d_3 = \frac{ \cos(3\pi) }{3} = -\frac{1}{3} \approx 0,33333 } \\
& \hresult{}{d_4 = \frac{ \cos(4\pi) }{4} = \frac{1}{4} = 0,25000 } \\
& \hresult{}{d_5 = \frac{ \cos(5\pi) }{5} = -\frac{1}{5} = 0,20000 0,008333 }
\end{align}
\end{subequations}

\sectionx{2}

\textbf{Para cada una de las siguientes sucesiones:}

\begin{enumerate}[(a)]

\bfseries

\item Encontrar el término 100 y el término 200 de cada una de ellas.

\item Hallar, si es posible, el término general $ a_n $.

\item Clasificar en convergentes o no convergentes.

\end{enumerate}

\begin{enumerate}[(i)]

\bfseries

\item $ 1, 2, 3, 4, \dots $

\item $ -1, -\frac{1}{2}, -\frac{1}{3}, -\frac{1}{4}, \dots $

\item $ 1, -\frac{1}{2}, \frac{1}{3}, -\frac{1}{4}, \dots $

\item $ \frac{1}{2}, -\frac{1}{4}, \frac{1}{8}, -\frac{1}{16}, \dots $

\item $ -1, 2, -3, 4, \dots $

\item $ 0, \frac{1}{2}, 0, \frac{1}{3}, 0, \frac{1}{4}, \dots $

\item $ 1, -1, 1, -1, \dots $

\item $ 2, \frac{3}{2}, \frac{4}{3}, \frac{5}{4}, \dots $

\item $ 1, 1, \frac{1}{2}, 2, \frac{1}{3}, 3, \frac{1}{4}, \dots $

\item $ a_1 = 1, a_{n+1} = 2a_n $

\end{enumerate}

\hrule

\subsectionx{2.i}

\begin{equation}
a_n = \{ 1, 2, 3, 4, \dots \}
\end{equation}

Esta es la sucesión de los números naturales. El término general no es otra cosa que $n$.

\begin{equation}
\hresult{2.i}{ a_n = n \Rightarrow a_{100} = 100, a_{200} = 200 }
\end{equation}

El conjunto $ \{a_n, n \in \mathbb{N}\} $ no está acotado, y por ende la sucesión es \textbf{no convergente}.

\subsectionx{2.ii}

\begin{equation}
a_n = \left\{ -1, -\frac{1}{2}, -\frac{1}{3}, -\frac{1}{4}, \dots \right\}
\end{equation}

Por inspección, el término general es:

\begin{equation}
\hresult{2.ii}{ a_n = -\frac{1}{n} \Rightarrow a_{100} = -\frac{1}{100} = 0,01, a_{200} = -\frac{1}{200} = -0,005 }
\end{equation}

Todos los elementos están entre -1 y 0, por lo que esta sucesión puede ser convergente. 

Por definición:

\begin{equation}
a_n \rightarrow L \Leftrightarrow \forall \epsilon > 0 \exists N / \forall n \geq N, |a_n - L| < \epsilon
\end{equation}

En palabras, la sucesión $ a_n $ converge a un número real $ L $ si y sólo si para todo $ \epsilon $ real positivo, es posible hallar un entero $ N $ tal que para todo $ n \geq N $, el módulo de $ a_n - L $ es menor a $ \epsilon $.

Para este caso particular, considérese primero la expresión $ |a_n - L| < \epsilon $. La sucesión parece converger a cero, por ende se plantea $ L = 0 $.

\begin{equation}
\left| \frac{1}{n} - 0 \right| < \epsilon \Rightarrow \frac{1}{n} < \epsilon \Rightarrow n > \frac{1}{e}
\end{equation}

Dado que al resolver la desigualdad para $ n $ se obtuvo $ n > \frac{1}{e} $, esto sugiere elegir $ N = \lceil \frac{1}{e} \rceil $.

El próximo paso es verificar que para esa elección de $ N $, la condición $ \frac{1}{n} < \epsilon $ se satisface para todo $ n \geq N $.

Si $ n \geq N \Rightarrow n \geq \lceil \frac{1}{\epsilon} \rceil $. Como son todos positivos, se puede invertir miembro a miembro, obteniendo:

\begin{equation}
\frac{1}{n} \leq \frac{1}{\lceil \frac{1}{\epsilon} \rceil }
\end{equation}

Quitar la función techo genera un numerador menor o igual, por ende es una cota superior válida.

\begin{equation}
\frac{1}{n} \leq \frac{1}{\lceil \frac{1}{\epsilon} \rceil } \geq \frac{1}{\frac{1}{\epsilon}} = \epsilon 
\end{equation}

Se demostró entonces que para esta elección de $ N $, $ |a_n - 0| $ está acotada para todo valor de $ \epsilon $. Por ende, \textbf{$ a_n $ converge a cero}.

\subsectionx{2.iii}

\begin{equation}
a_n = \left\{ 1, -\frac{1}{2}, \frac{1}{3}, -\frac{1}{4}, \dots \right\}
\end{equation}

Es la misma sucesión del inciso anterior, pero sólo los términos de índice par son negativos. Para lograr esto, se puede introducir un factor $ (-1)^{n+1} $. Nótese que se introdujo un +1 porque sin eso, los términos de índice impar quedarían negativos, y se desea hacer eso para los de índice par.

\begin{equation}
\hresult{2.iii}{ a_n = (-1)^{n+1} \frac{1}{n} \Rightarrow a_{100} = -\frac{1}{100} = -0,01, a_{200} = -\frac{1}{200} = -0,005 }
\end{equation}

Al tomar el módulo, esta sucesión se convierte en $ \frac{1}{n} $, por ende \textbf{converge a cero}.

\subsectionx{2.iv}

\begin{equation}
a_n = \left\{ \frac{1}{2}, -\frac{1}{4}, \frac{1}{8}, -\frac{1}{16}, \dots \right\}
\end{equation}

Al tener signo alternado, de nuevo hay un factor $ (-1)^{n+1} $, ya que los negativos son los de índice par. El numerador siempre es 1, y el denominador es la sucesión de potencias de 2.

\begin{equation}
\hresult{2.iv}{ a_n = (-1)^{n+1} \frac{1}{2^n} \Rightarrow a_{100} = -\frac{1}{2^{100}}, a_{200} = -\frac{1}{2^{200}} }
\end{equation}

En cuanto a la convergencia, al tomar módulo se tiene $ \frac{1}{2^n} $, que es término a término menor a $ \frac{1}{n} $. Esto permite utilizar la propiedad de que si una sucesión es menor término a término respecto a una sucesión convergente, entonces también converge. Por ende, $ a_n $ \textbf{converge a cero}.

\subsectionx{2.v}

\begin{equation}
a_n = \{ -1, 2, -3, 4, \dots \}
\end{equation}

Es la sucesión de los naturales con un factor $ (-1)^n $. Como en este caso los negativos son los términos de índice impar, se usa $ (-1)^n $ en lugar de $ (-1)^{n+1} $.

\begin{equation}
\hresult{2.v}{ a_n = (-1)^n n \Rightarrow a_{100} = 100, a_{200} = 200 }
\end{equation}

Esta sucesión no está acotada, y por ende \textbf{no converge}.

\subsectionx{2.vi}

\begin{equation}
a_n = \left\{ 0, \frac{1}{2}, 0, \frac{1}{3}, 0, \frac{1}{4}, \dots \right\}
\end{equation}

Cuando se tiene una sucesión con una forma clara para términos de índice impar y otra forma diferente para los de índice par, suele ser más práctico expresarla de la siguiente manera:

\begin{subequations}
\begin{align}
& \hresult{2.vi}{ a_{2n-1} = 0 } \\
& \hresult{2.vi}{ a_{2n} = \frac{1}{n} } \\
& \hresult{2.vi}{ a_{100} = \frac{1}{50}, a_{200} = \frac{1}{100} }
\end{align}
\end{subequations}

Nótese que el término general tiene una forma para los términos de índice par, $ a_{2n} $, y otra para los términos de índice impar: $ a_{2n-1} $.

Para analizar la convergencia, se utiliza la propiedad de que si una sucesión converge, todas sus subsucesiones también, y al mismo valor. En este caso, una es una constante y la otra converge a cero. Dado que la constante coincide con cero, podemos afirmar que la sucesión completa \textbf{converge a cero}.

\subsectionx{2.vii}

\begin{equation}
a_n = \{ 1, -1, 1, -1, \dots \}
\end{equation}

Es $ (-1)^{n+1} $, ya que los negativos son los de índice par.

\begin{equation}
\hresult{2.viii}{ a_n = (-1)^{n+1} \Rightarrow a_{100} = a_{200} = -1 }
\end{equation}

Esta sucesión está acotada, pero no converge. Si se toma la subsucesión de índices pares, es la constante -1. Y si se toma la subsucesión de índices impares, es la constante 1. Es en este punto donde se aplica la propiedad de que si dos subsucesiones convergen a distintos valores, la sucesión completa \textbf{no converge}.

\subsectionx{2.viii}

\begin{equation}
a_n = \left\{ 2, \frac{3}{2}, \frac{4}{3}, \frac{5}{4}, \dots \right\}
\end{equation}

Por inspección, el numerador es $ n+1 $ y el denominador es $ n $.

\begin{equation}
a_n = \frac{n+1}{n} \Rightarrow a_{100} = \frac{101}{100}, a_{200} = \frac{201}{200}
\end{equation}

Para valores grandes de $ n $, el +1 del numerador es despreciable, por lo que queda $ \frac{n}{n} $ y puede intuirse que esta sucesión converge a 1. Aplicando la definición, considérese:

\begin{subequations}
\begin{align}
& \left| \frac{n+1}{n} - 1 \right| < \epsilon \\
& \left| \frac{n+1-n}{n} \right| < \epsilon \\
& \frac{1}{n} < \epsilon
\end{align}
\end{subequations}

No es necesario continuar, porque demostrar esta convergencia es equivalente a demostrar la de $ \frac{1}{n} $, y eso ya se hizo. Conclusión: la sucesión \textbf{converge a 1}.

\subsectionx{2.ix}

\begin{equation}
a_n = \left\{ 1, 1, \frac{1}{2}, 2, \frac{1}{3}, 3, \frac{1}{4}, \dots \right\}
\end{equation}

Nuevamente, hay una subsucesión para los índices pares y otra para los impares. Una forma cómoda de expresar esto es la ya vista:

\begin{subequations}
\begin{align}
& \hresult{2.ix}{ a_{2n-1} = n } \\
& \hresult{2.ix}{ a_{2n} = \frac{1}{n} } \\
& \hresult{2.ix}{ a_{100} = \frac{1}{50}, a_{200} = \frac{1}{100} }
\end{align}
\end{subequations}

En cuanto a la convergencia, se utilizará la propiedad de que si una subsucesión no converge, la sucesión completa no converge. En este caso, la sucesión de índices impares no converge, por ende no hace falta analizar la otra para concluir que la sucesión completa \textbf{no converge}.

\subsectionx{2.x}

Observando los valores de los primeros términos, se obtiene:

\begin{equation}
a_n = \{ 1, 2, 4, 8, 16, \dots... \}
\end{equation}

Son las potencias de dos comenzando en el exponente cero. Por lo tanto:

\begin{equation}
\hresult{2.x}{ a_n = 2^{n-1} \Rightarrow a_{100} = 2^{99}, a_{200} = 2^{199} }
\end{equation}

Esta sucesión no está acotada, por lo cual \textbf{no converge}.

\sectionx{3}

\textbf{Hallar un valor de $ n \in \mathbb{N} $ a partir del cual haya certeza de que: }

\begin{enumerate}[(a)]

\bfseries

\item $ n^2 - 5n - 8 $ sea mayor que \textbf{(i)} 10 y \textbf{(ii)} 1000.

\item $ 2^n - 100 $ sea mayor que \textbf{(i)} 10 y \textbf{(ii)} 1000.

\item $ \frac{(-1)^n}{n+1} + 2 $ esté entre \textbf{(i)} 1,9 y 2,1 y \textbf{(ii)} 1,999 y 2,001.

\item $ \frac{\sin n}{n} $ esté entre \textbf{(i)} -0,1 y 0,1 y \textbf{(ii)} -0,001 y 0,001.  

\end{enumerate}

\subsectionx{3.a}

Sea el caso general:

\begin{equation}
n^2 - 5n -8 > k \Leftrightarrow n^2 - 5n -8 - k > 0
\end{equation}

En este punto, conviene expresar la función cuadrática en forma factorizada según sus raíces $ x_1 $ y $ x_2 $. Nótese que para este análisis se usa una función de variable continua y no una sucesión, pero dado que el análisis vale para todo $ x $ real, vale para el subconjunto de los valores de la sucesión.

\begin{equation}
x^2 - 5x -8 - k = (x-x_1) (x-x_2)
\end{equation}

Aplicando la fórmula de las raíces:

\begin{equation}
x_1, x_2 = \frac{-b \pm \sqrt{b^2 - 4 a c}}{2a} = \frac{5 \pm \sqrt{25 - 4 (-8-k)}}{2} = \frac{5 \pm \sqrt{57 + 4k}}{2}
\end{equation}

Para k = 10, resulta:

\begin{subequations}
\begin{align}
& x_1 = \frac{5+\sqrt{97}}{2} \approx 7,4244 \\
& x_2 = \frac{5-\sqrt{97}}{2} \approx -2,4244
\end{align}
\end{subequations}

Dado que $ f(x) = (x-x_1) (x-x_2) $ y se está buscando los valores de $ x $ para los que $ f(x) $ es positiva, hay dos escenarios.

Caso 1: ambos factores del producto positivos.

\begin{equation}
x > x_1 \wedge x > x_2 \Rightarrow x > \mathop{\text{max}}(x_1, x_2) \Rightarrow x > 7,4244
\end{equation}

Caso 2: ambos factores del producto negativos.

\begin{equation}
x < x_1 \wedge x < x_2 \Rightarrow x < \mathop{\text{min}}(x_1, x_2) \Rightarrow x < -2,4244
\end{equation}

Volviendo a los números naturales, el intervalo negativo no interesa. Y para el caso positivo, el primer entero mayor a $ x_1 $ es 8. Por ende, se obtiene:

\begin{equation}
\hresult{3.a.i}{ k = 10 \Rightarrow n \geq 8 }
\end{equation}

Repitiendo este análisis para $ k = 1000 $, se obtiene:

\begin{equation}
\hresult{3.a.ii}{ k = 1000 \Rightarrow n \geq 35 }
\end{equation}

\subsectionx{3.b}

Nuevamente, analizando de manera general:

\begin{subequations}
\begin{align}
& 2^n - 100 > k \\
& 2^n > k + 100 \\
& \log_2( 2^n ) > \log_2( k + 100 ) \\
& n > \log_2( k + 100 )
\end{align}
\end{subequations}

Para los valores solicitados de $ k $, resulta:

\begin{subequations}
\begin{align}
& \hresult{3.b.i}{ k = 10 \Rightarrow n \geq 7 } \\
& \hresult{3.b.ii}{ k = 1000 \Rightarrow n \geq 11 }
\end{align}
\end{subequations}

\subsectionx{3.c}

En este caso hay una cota superior y otra inferior, lo cual dificulta el análisis genérico. Para el primer caso:

\begin{subequations}
\begin{align}
& 1,9 < \frac{(-1)^n}{n+1} + 2 < 2,1 \\
& -0,1 < \frac{(-1)^n}{n+1} < 0,1
\end{align}
\end{subequations}

Llegado este punto, nótese que las cotas son el mismo valor pero con distinto signo. Por lo tanto, las desigualdades pueden simplificarse en una sola utilizando la siguiente propiedad de la función módulo:

\begin{equation}
|x| < a, a > 0 \Leftrightarrow -a < x < a
\end{equation}

Para este caso, se obtiene:

\begin{subequations}
\begin{align}
& \left| \frac{(-1)^n}{n+1} \right| < 0,1 \\
& \frac{1}{n+1} < 0,1 \\
& 1 < 0,1 (n+1) \\
& \hresult{3.c.i}{ n \geq 10 }
\end{align}
\end{subequations}

Haciendo el mismo análisis para el segundo caso:

\begin{equation}
\hresult{3.c.ii}{ n \geq 1000 }
\end{equation}

\subsectionx{3.d}

Aunque de nuevo haya dos desigualdades, es posible explotar las propiedades de la función seno para hacer un análisis genérico.

\begin{subequations}
\begin{align}
& -k < \frac{\sin n}{n} < k \\
& -k n < \sin n < k n
\end{align}
\end{subequations}

La función seno está acotada entre -1 y 1, por ende debe satisfacerse:

\begin{subequations}
\begin{align}
& -k n \leq -1 \Rightarrow n \geq \frac{1}{k} \\
& k n \geq 1 \Rightarrow n \geq \frac{1}{k}
\end{align}
\end{subequations}

Finalmente:

\begin{subequations}
\begin{align}
& \hresult{3.d.i}{ k = 0,1 \Rightarrow n \geq 10 } \\
& \hresult{3.d.ii}{ k = 0,001 \Rightarrow n \geq 1000 }
\end{align}
\end{subequations}

\sectionx{4}

\textbf{ Considerar la sucesión $ a_n = \frac{n+1}{n-1000,2} $. A partir de que $ \limninf a_n = 1 $, responder cuáles de las siguientes afirmaciones son verdaderas, justificando en cada caso. }

\begin{enumerate}[(a)]

\bfseries

\item Existe un $ n \in \mathbb{N} $ a partir del cual $ a_n > 0 $.

\item Existe un $ n \in \mathbb{N} $ a partir del cual $ a_n > \frac{1}{2} $.

\item Existe un $ n \in \mathbb{N} $ a partir del cual $ a_n < 1 $.

\item Existe un $ n \in \mathbb{N} $ para el cual $ a_n = 1 $.

\item La sucesión $ a_n $ está acotada.

\end{enumerate}

\textbf{ Escribir las afirmaciones que correspondan con la nomenclatura pctn. }

\vspace{1em}
\hrule
\vspace{1em}

La nomenclatura pctn significa ``para casi todo n'', y se usa para indicar propiedades que se cumplen para todos los valores naturales excepto una cantidad finita de valores. Por ejemplo, se puede decir que la sucesión $ a_n = n-3 $ es mayor a cero pctn, dado que sólo es negativa o cero para $ n = 1, n=2 $ y $n=3$.

De manera más formal y general, un predicado $P(n)$, donde $P:\mathbb{N} \rightarrow {F, V}$ se cumple pctn si y sólo el conjunto $ \{ P(n), n \in \mathbb{N} / P(n) = F  \} $ es acotado.

\subsectionx{4.a}

Existe un $ n \in \mathbb{N} $ a partir del cual $ a_n > 0 $.

Esta afirmación es equivalente a decir que $ a_n > 0 $ pctn. A priori, cabe esperar que esto sea \textbf{verdadero}, ya que al ser 1 el límite, a partir de cierto $ n $ los valores estarán concentrados alrededor de 1. Una forma de verificar esto es ver para qué valores de $ n $ resulta $ a_n $ positiva.

\begin{equation}
\frac{n+1}{n-1000, 2} > 0
\end{equation}

El numerador es siempre positivo, por ser $ n $ natural. Por lo tanto, para que el cociente sea positivo, debe ser el denominador positivo también. Ergo:

\begin{equation}
n-1000,2 > 0 \Rightarrow n > 1000,2 \Rightarrow n \geq 1001
\end{equation}

\begin{equation}
\hresult{4.a}{ \textbf{(V)} a_n > 0 \text{ pctn, dado que } a_n > 0 \text{ para } n \geq 1001 }
\end{equation}

\subsectionx{4.b}

Existe un $ n \in \mathbb{N} $ a partir del cual $ a_n > \frac{1}{2} $.

De manera análoga al inciso anterior, al ser el límite 1, cabe esperar que haya infinitos valores de $ a_n $ en la vecindad de 1. para confirmar esto, se analizará para qué valores de $ n $ se cumple que $ a_n $ es mayor a $ \frac{1}{2} $.

\begin{subequations}
\begin{align}
& \frac{n+1}{n-1000,2} > \frac{1}{2} \\
& \frac{2(n+1)}{n-1000,2} > 1 \\
& \frac{2n+2}{n-1000,2} - 1 > 0 \\
& \frac{2n+2-(n-1000,2)}{n-1000,2} > 0 \\
& \frac{n+1002,2}{n-1000,2} > 0
\end{align}
\end{subequations}

El numerador siempre es positivo, por ende el denominador tiene que serlo. Esto conduce nuevamente a $ a_n \geq 1001 $. Ergo:

\begin{equation}
\hresult{4.b}{ \textbf{(V)} a_n > \frac{1}{2} \text{ pctn, dado que } a_n > \frac{1}{2} \text{ para } n \geq 1001 }
\end{equation}

\subsectionx{4.c}

Existe un $ n \in \mathbb{N} $ a partir del cual $ a_n < 1 $.

En este caso, cabe esperar que la afirmación sea falsa, ya que al ser el numerador siempre mayor que el denominador, el cociente $ a_n $ tiende a 1 ``por arriba''. Formalmente:

\begin{subequations}
\begin{align}
& \frac{n+1}{n-1000,2} < 1 \\
& \frac{n+1}{n-1000,2} - 1 < 0 \\
& \frac{n+1-(n-1000,2)}{n-1000,2} < 0 \\
& \frac{1001,2}{n-1000,2} < 0 \\
& n < 1000,2 \\
& n \leq 1000
\end{align}
\end{subequations}

\begin{equation}
\hresult{4.c}{ \textbf{(F)} a_n < 1 \text{ sólo para } n \leq 1000 (a_n \geq 1 \text{ pctn}) }
\end{equation}

\subsectionx{4.d}

Existe un $ n \in \mathbb{N} $ para el cual $ a_n = 1 $.

En este caso, el límite nunca se alcanza, ni en el infinito ni en valores particulares. Téngase en cuenta que sí puede ocurrir que el límite sea 1 y el valor 1 se alcance. Por ejemplo, para la sucesión constante $ a_n = 1 $. Pero volviendo a este caso:

\begin{subequations}
\begin{align}
& \frac{n+1}{n-1000,2} = 1 \\
& n+1 = n-1000,2 \\
& 1 = -1000,2
\end{align}
\end{subequations}

\begin{equation}
\hresult{4.d}{ \textbf{(F)} \text{ No existe valor de } n \text{ natural que satisfaga } a_n = 1 }
\end{equation}

\subsectionx{4.e}

La sucesión $ a_n $ está acotada.

Que una sucesión esté acotada es una condición necesaria para que sea convergente. Por ende, que el límite exista y sea finito implica que $ a_n $ está acotada.

\begin{equation}
\hresult{4.e}{ (V) a_n \text{ convergente } \Rightarrow a_n \text{ acotada. } }
\end{equation}

\sectionx{5}

\textbf{Calcular, si existe, el límite de las siguientes sucesiones. En cada caso, explicar las propiedades utilizadas para obtener la respuesta.}

\begin{enumerate}[(a)]

\bfseries

\item $ a_n = \frac{-4n^3 + 2n^2 - 3n -1}{5n^2 + 4} $

\item $ a_n = \frac{7n^3-5}{n+3} $

\item $ a_n = \frac{\sqrt{n^3}+2}{n^2-1} $

\item $ a_n = \sqrt{ \frac{2n^2-1}{3n^2+2} } $

\item $ a_n = \frac{4n^2 + 3}{3n^2 + 4000} $

\item $ a_n = \frac{-n}{\sqrt{n^2-n}+n} $

\item $ a_n = \{ 3, \frac{4}{3}, \frac{5}{2}, \frac{8}{5}, \frac{7}{3}, \frac{16}{9}, \frac{9}{4}, \frac{32}{17}, \frac{11}{5}, \frac{64}{33}, \dots \} $

\item $ a_n = \{ 1, 1 + \frac{1}{2}, 1 + \frac{1}{2} + \frac{1}{4}, \dots \} $

\end{enumerate}

\hrule

\subsectionx{5.a}

\begin{equation}
a_n = \frac{-4n^3 + 2n^2 - 3n -1}{5n^2 + 4}
\end{equation}

Al evaluar expresiones polinómicas en el infinito, se descartan todos los términos excepto el de mayor grado. Esto es porque al estar asociado a la potencia más alta, para valores grandes de $ n $, los demás términos serán despreciables en comparación. Para este caso:

\begin{equation}
\limninf \frac{-4n^3 + 2n^2 - 3n -1}{5n^2 + 4} = \limninf \frac{-4n^3}{5n^2} = \limninf -\frac{4}{5} n
\end{equation}

Se adoptará la convención de que los límites infinitos no existen, y se informará hacia dónde diverge la sucesión en dichos casos.

\begin{equation}
\hresult{5.a}{ \limninf a_n \text{ no existe, y } a_n \text{ diverge a } -\infty }
\end{equation}

\subsectionx{5.b}

\begin{equation}
a_n = \frac{7n^3-5}{n+3}
\end{equation}

\begin{equation}
\limninf \frac{7n^3-5}{n+3} = \limninf \frac{7n^3}{n} = \limninf 7n^2
\end{equation}

\begin{equation}
\hresult{5.a}{ \limninf a_n \text{ no existe, y } a_n \text{ diverge a } +\infty }
\end{equation}

\subsectionx{5.c}

\begin{equation}
a_n = \frac{\sqrt{n^3}+2}{n^2-1}
\end{equation}

Estrictamente hablando, el numerador no es una expresión polinómica porque hay un exponente fraccionario. De todas maneras, de manera más general, si se tiene una suma de potencias, al evaluar en el infinito, se puede descartar todos los términos excepto el de mayor exponente. En este caso:

\begin{equation}
\limninf \frac{\sqrt{n^3}+2}{n^2-1} = \limninf \frac{n^{\frac{3}{2}} +2}{n^2-1} = \limninf \frac{n^{\frac{3}{2}}}{n^2} = \limninf n^{-\frac{1}{2}} = \limninf \sqrt{\frac{1}{n}}
\end{equation}

Ya se ha demostrado que $ \limninf \frac{1}{n} = 0	 $. Al tomar la raíz cuadrada, se tiene una sucesión que es término a término menor que $ \frac{1}{n} $. Por ende, es convergente, y en este caso al mismo valor.

\begin{equation}
\hresult{5.c}{\limninf a_n = 0}
\end{equation}

\subsectionx{5.d}

\begin{equation}
a_n = \sqrt{ \frac{2n^2-1}{3n^2+2} }
\end{equation}

De manera similar al caso anterior, aunque no se tenga un cociente de polinomios, el argumento de la raíz cuadrada sí lo es y pueden despreciarse los términos de menor grado.

\begin{equation}
\limninf \sqrt{ \frac{2n^2-1}{3n^2+2} } = \limninf \sqrt{ \frac{2n^2}{3n^2} } = \limninf \sqrt{\frac{2}{3}} = \sqrt{\frac{2}{3}} = \sqrt{\frac{2}{3}} \frac{\sqrt{3}}{\sqrt{3}} = \frac{\sqrt{6}}{3}
\end{equation}

\begin{equation}
\hresult{5.d}{\limninf a_n = \frac{\sqrt{6}}{3}}
\end{equation}

\subsectionx{5.e}

\begin{equation}
a_n = \frac{4n^2 + 3}{3n^2 + 4000}
\end{equation}

\begin{equation}
\limninf \frac{4n^2 + 3}{3n^2 + 4000} = \limninf \frac{4n^2}{3n^2} = \frac{4}{3}
\end{equation}

\begin{equation}
\hresult{5.e}{\limninf a_n = \frac{4}{3}}
\end{equation}

\subsectionx{5.f}

\begin{equation}
a_n = \frac{-n}{\sqrt{n^2-n}+n}
\end{equation}

Esta expresión tampoco es un polinomio. Pero dentro de la raíz cuadrada, sí hay un polinomio. Para valores muy grandes de $ n $, $ n $ es despreciable comparado con $ n^2 $. Por lo tanto:

\begin{equation}
\limninf \frac{-n}{\sqrt{n^2-n}+n} = \limninf \frac{-n}{\sqrt{n^2}+n}
\end{equation}

Siendo n natural, es siempre positivo y por ende vale cancelar la raíz con el cuadrado de manera directa.

\begin{equation}
\limninf \frac{-n}{\sqrt{n^2}+n} = \limninf \frac{-n}{n+n} = \limninf \frac{-n}{2n} = -\frac{1}{2}
\end{equation}

\begin{equation}
\hresult{5.f}{\limninf a_n = -\frac{1}{2}}
\end{equation}

\subsectionx{5.g}

\begin{equation}
a_n = \left\{ 3, \frac{4}{3}, \frac{5}{2}, \frac{8}{5}, \frac{7}{3}, \frac{16}{9}, \frac{9}{4}, \frac{32}{17}, \frac{11}{5}, \frac{64}{33}, \dots \right\}
\end{equation}

Hay una subsucesión para los índices impares, y otra para los pares. Por observación, la primera tiene los números impares empezando en 3 el numerador, y $ n $ en el denominador.

\begin{equation}
a_{2n-1} = \frac{2n+1}{n}
\end{equation}

En cuanto a la sucesión de índices pares, el numerador es siempre una potencia de dos, empezando en 4. Y el denominador es la potencia anterior a la del numerador, sumándole 1.

\begin{equation}
a_{2n} = \frac{2^{n+1}}{2^n + 1}
\end{equation}

Una propiedad ya mencionada es que si una sucesión converge a un valor, todas sus subsucesiones convergen a ese mismo valor. Recíprocamente, si se consideran subsucesiones que conforman la sucesión completa, como es el caso de índices pares e impares, si ambas convergen a un mismo valor, ése será necesariamente el límite de la sucesión completa. En este caso particular:

\begin{equation}
\limninf a_{2n-1} = \limninf \frac{2n+1}{n} = \limninf \frac{2n}{n} = 2
\end{equation}

\begin{equation}
\limninf a_{2n} = \limninf \frac{2^{n+1}}{2^n+1} = \limninf \frac{2^{n+1}}{2^n} = 2
\end{equation}

Nótese que en el cálculo del segundo límite se descartó el 1 del denominador porque es despreciable comparado con $ 2^n $. Finalmente, resulta:

\begin{equation}
\hresult{5.g}{\limninf a_n = 2}
\end{equation}

\subsectionx{5.h}

\begin{equation}
a_n = \left\{ 1, 1 + \frac{1}{2}, 1 + \frac{1}{2} + \frac{1}{4}, \dots \right\}
\end{equation}

Calculando los primeros 5 términos, es posible inferir el término general de esta sucesión.

\begin{equation}
a_n = \left\{ 1, \frac{3}{2}, \frac{7}{4}, \frac{15}{8}, \frac{31}{16}, \dots \right\}
\end{equation}

El numerador es la n-ésima potencia de 2 menos uno, y el denominador es la potencia de 2 anterior.

\begin{equation}
a_n = \frac{2^n-1}{2^{n-1}} \Rightarrow \limninf a_n = \limninf \frac{2^n}{2^{n-1}} = 2
\end{equation}

Como solución alternativa, también es posible aplicar la siguiente propiedad. La suma de una serie geométrica con razón $ r $ es, para $ |r| < 1 $:

\begin{equation}
|r| < 1 \Rightarrow \sum_{k=0}^{+\infty} a \cdot r^k = \frac{a}{1-r} 
\end{equation}

Para este caso, $ a = 1 $, $ r = \frac{1}{2} $, lo cual conduce a que la suma vale 2.

\begin{equation}
\hresult{5.h}{\limninf a_n = 2}
\end{equation}

\end{document}
