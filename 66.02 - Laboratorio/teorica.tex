\documentclass{article}

\usepackage[spanish]{babel}
\usepackage[margin=1.5in]{geometry}
\usepackage[utf8]{inputenc}

\begin{document}

\setcounter{section}{-1}
\section{Conceptos básicos de redes eléctricas}

\subsection{Magnitudes eléctricas}

\begin{itemize}
	\item Intensidad de corriente eléctrica: Cantidad de carga que atraviesa un conductor por unidad de tiempo. Unidad: Ampère / Amperio (A). Órdenes usuales de magnitud (O.U.M.): $\mu$A, mA
	\item Diferencia de potencial (entre dos puntos): Causa/origen del paso de una corriente eléctrica a través
	de un conductor. Unidad: Volt/Voltio (V). O.U.M.: mV, V
	\item Frecuencia: Esta magnitud se verá más en detalle al ver corriente alterna. Unidad: Hertz (Hz).
	 O.U.M.: Hz, kHz, MHz.
\end{itemize}

\end{document} 